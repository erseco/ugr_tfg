\chapter{Glosario de términos}


\textbf{OPQA}: En los primeros ordenadores que aparecieron en España en la década de los 80 esta era la combinación de teclas que tenían predefinidos la mayoría de los juegos siendo OP las teclas de izquierda y derecha y QA las de arriba y abajo.
\bigskip

\textbf{Usabilidad}: facilidad con que las personas pueden utilizar una herramienta particular o cualquier otro objeto fabricado por humanos con el fin de alcanzar un objetivo concreto. La usabilidad es un término que no forma parte del diccionario de la Real Academia Española (RAE), aunque es bastante habitual en el ámbito de la informática y la tecnología.
\bigskip

\textbf{Accesibilidad}: grado en el que todas las personas pueden utilizar un objeto, visitar un lugar o acceder a un servicio, independientemente de sus capacidades técnicas, cognitivas o físicas. Es indispensable e imprescindible, ya que se trata de una condición necesaria para la participación de todas las personas independientemente de las posibles limitaciones funcionales que puedan tener.
\bigskip

\textbf{Seguridad de la información}: conjunto de medidas preventivas y reactivas de las organizaciones y de los sistemas tecnológicos que permiten resguardar y proteger la información buscando mantener la confidencialidad, la disponibilidad e integridad de datos y de la misma.
\bigskip

\textbf{Disponibilidad}: medida que nos indica cuánto tiempo está disponible ese equipo o sistema operativo respecto de la duración total durante la que se hubiese deseado que funcionase.
\bigskip

 \textbf{Auditoría de seguridad}: estudio que comprende el análisis y gestión de sistemas llevado a cabo por profesionales para identificar, enumerar y posteriormente describir las diversas vulnerabilidades que pudieran presentarse en una revisión exhaustiva de las estaciones de trabajo, redes de comunicaciones o servidores.
\bigskip

 \textbf{WCAG}: Pautas de Accesibilidad para el Contenido Web (WCAG) 2.0 (traducción al castellano de Web Content Accessibility Guidelines 2.0) son una serie de normas de accesibilidad Web definidas por el World Wide Web Consortium (W3C).
\bigskip




\textbf{SCORM}: (del inglés Sharable Content Object Reference Model) es un conjunto de estándares y especificaciones que permite crear objetos pedagógicos estructurados. Los sistemas de gestión de contenidos en web originales usaban formatos propietarios para los contenidos que distribuían.
\bigskip

\textbf{Mozilla Open Badges}: es un programa de la Fundación Mozilla con el que conseguir reconocimiento digital por las cualidades y logros que obtenemos fuera del ámbito académico, facilitando que quien lo desee pueda emitir, ganar y mostrar insignias en la web a través de una infraestructura técnica.
\bigskip

\textbf{Blog}: es una contracción de web log. Los blogs son una forma de revista (journal) en línea usada por millones de personas en el mundo para expresarse a sí mismas y comunicarse con familiares y amigos.
\bigskip

\textbf{Fundación Mozilla}: organización sin ánimo de lucro que produce software libre.
\bigskip

\textbf{Frontend}: es la interfaz de la aplicación, es la parte de la aplicación que el usuario utiliza para comunicarse con la misma.
\bigskip

\textbf{Backend}: es el motor de una aplicación, se encarga de realizar las funciones en segundo plano que se encargan de que la aplicación funcione.
\bigskip

\textbf{URL (Uniform Resource Locator)}: nombre y con un formato estándar que permite acceder a un recurso de forma inequívoca.
\bigskip

\textbf{HTML (HyperText Markup Language)}: lenguaje de marcado que se utiliza para la realización de páginas web.
\bigskip

\textbf{JavaScript}: lenguaje de programación orientado a objetos interpretado que se utiliza principalmente para cargar programas desde el lado del cliente en los navegadores web.
\bigskip

\textbf{JSON (JavaScript Object Notation)}: formato de texto plano usado para el intercambio de información, independientemente del lenguaje de programación.
\bigskip

\textbf{ettercap}: Ettercap es un sniffer. Soporta direcciones activas y pasivas de varios protocolos (incluso aquellos cifrados, como SSH y HTTPS). También hace posible la inyección de datos en una conexión establecida y filtrado al vuelo aun manteniendo la conexión sincronizada gracias a su poder para establecer un Ataque Man-in-the-middle(Spoofing).
\bigskip

\textbf{Expresión regular}: Una expresión regular, a menudo llamada también regex, es una secuencia de caracteres que forma un patrón de búsqueda, principalmente utilizada para la búsqueda de patrones de cadenas de caracteres u operaciones de sustituciones
\bigskip

\textbf{Exploit}: (del inglés exploit, 'explotar' o 'aprovechar') es un fragmento de software, fragmento de datos o secuencia de comandos y/o acciones, utilizada con el fin de aprovechar una vulnerabilidad de seguridad de un sistema de información para conseguir un comportamiento no deseado del mismo.
\bigskip

\textbf{GreaseMonkey}: Extensión que incorporan algunos navegadores web para poder inyectar código Javascript en un determinado sitio web para modificar su apariencia y/o comportamiento.
\bigskip

\textbf{LaTeX}: sistema de composición de documentos que permite crear textos en diferentes formatos (artículos, cartas, libros, informes...) obteniendo una alta calidad en los documentos generados.
\bigskip


\textbf{SSH (Secure SHell)}: protocolo que permite conectarse a máquinas remotas mediante conexiones seguras de red.
\bigskip

\textbf{RSA}: sistema criptográfico de clave pública usado para la seguridad de transferencia de datos.
\bigskip

\textbf{SSL (Secure Sockets Layer)}: serie de protocolos criptográficos que proporcionan comunicaciones seguras por una red.
\bigskip


\textbf{iptables}: Iptables es un poderoso firewall integrado en el kernel de Linux y que forma parte del proyecto netfilter.
\bigskip



\textbf{WireShark}: Es un analizador de protocolos utilizado para realizar análisis y solucionar problemas en redes de comunicaciones, para desarrollo de software y protocolos, y como una herramienta didáctica.
\bigskip

\textbf{StatusCake}: Es un servicio web que permite monitorizar la disponibilidad de servicios online.
\bigskip

\textbf{R}: Entorno y lenguaje de programación con un enfoque al análisis estadístico. Se trata de uno de los lenguajes más utilizados en investigación por la comunidad estadística, siendo además muy popular en el campo de la minería de datos, la investigación biomédica, la bioinformática y las matemáticas financieras. 
\bigskip

\textbf{Google Forms}: Google Forms es una herramienta que permite recopilar información de los usuarios a través de una encuesta o un cuestionario personalizado. La información se recoge y se conecta automáticamente a una hoja de cálculo. La hoja de cálculo se rellena con las respuestas de encuesta y cuestionario.
\bigskip

\textbf{Google SpreadSheet}: Es un servicio vía web de hojas de cálculo, realizado en tecnología AJAX.
\bigskip



\textbf{YUI}: Yahoo User Interface(YUI), una serie de bibliotecas escritas en JavaScript, para la construcción de aplicaciones interactivas cuyo desarrollo se descontinuó el 29 de agosto de 2014



\textbf{Módulo}: fragmento de un programa desarrollado para realizar una tarea específica.
\bigskip

\textbf{Balanceo de carga}: técnica de configuración de servidores que permite que la carga de trabajo total se reparte entre varios de ellos para que no disminuya el rendimiento general de la infraestructura.
\bigskip


\textbf{MySQL}: sistema de gestión de bases de datos relacional desarrollado bajo licencia dual GPL/Licencia comercial por Oracle Corporation.
\bigskip

\textbf{Hash}: también llamadas funciones de resumen son algoritmos que consiguen crear a partir de una entrada (ya sea un texto, una contraseña o un archivo, por ejemplo) una salida alfanumérica de longitud normalmente fija que representa un resumen de toda la información que se le ha dado (es decir, a partir de los datos de la entrada crea una cadena que solo puede volverse a crear con esos mismos datos).
\bigskip

\textbf{Honeypot}: Un honeypot, o sistema trampa, es una herramienta de la seguridad informática dispuesto en una red o sistema informático para ser el objetivo de un posible ataque informático, y así poder detectarlo y obtener información del mismo y del atacante.
\bigskip

\textbf{Oracle Database}: es un sistema de gestión de base de datos de tipo objeto-relacional, desarrollado por Oracle Corporation.
\bigskip

 \textbf{Objeto-Relacional}: es una extensión de la base de datos relacional tradicional, a la cual se le proporcionan características de la programación orientada a objetos.
\bigskip

\textbf{Software libre}: software cuya licencia permite que este sea usado, copiado, modificado y distribuido libremente según el tipo de licencia que adopte.
\bigskip
