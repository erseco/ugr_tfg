\begin{center}
{\LARGE\bfseries\titulo}\\
\end{center}
\begin{center}
\autor\
\end{center}

\section*{Resumen}

\bigskip
\noindent{\textbf{Palabras clave}: \textit{\keywords}\\

Este proyecto tiene como objetivo la elaboración de un informe profesional sobre la \textit{Plataforma de Recursos de Apoyo a la Docencia} en uso actualmente por la \textit{Universidad de Granada} analizando su {\tt Accesibilidad}, {\tt Usabilidad}, {\tt Seguridad} y {\tt Disponibilidad}.

\bigskip
Para dicho análisis se realizaron encuestas a usuarios de la plataforma, se analizaron las estadísticas de uso en colaboración con los administradores de la misma, se analizó el código de la plataforma \textit{moodle}, se revisaron las recomendaciones de buenas practicas de sus desarrolladores y se estudiaron algunas de las vulnerabilidades documentadas en sus CVE\footnote{Common Vulnerabilities and Exposures, siglas CVE, es una lista de información registrada sobre conocidas vulnerabilidades de seguridad, donde cada referencia tiene un número de identificación único.}.

\bigskip
El informe final tiene como objetivo dar a conocer tanto las fortalezas como las debilidades de la plataforma Prado2 con objeto de servir de ayuda a sus administradores para mejorar la misma y ayudándoles a su vez a conocer la opinión de sus usuarios.


\newpage
\begin{center}
{\LARGE\bfseries\tituloEng}\\
\end{center}
\begin{center}
\autor\
\end{center}

\section*{Extended abstract}

\bigskip
\noindent{\textbf{Keywords}: \textit{\keywordsen}.\\

The goal of this project is to develop a professional report on the \textit{Plataforma de Recursos de Apoyo a la Docencia} (Teaching Support Resources Platform), currently in use by the \textit{University of Granada}, analyzing its accessibility, usability, security and availability.

\bigskip
To develop this report, we have analyzed the data comming from surveys made by platform's users and usage statistics. The usage statistics have been analyzed in cooperation with its administrators. Also, we have analyzed Moodle's source code, reviewed the \textit{Moodle developers' good practices recommendations} and studied some of the vulnerabilities documented in the \textit{Common Vulnerabilities and Exposures} or \textit{CVE}\footnote{The Common Vulnerabilities and Exposures (CVE) system provides a reference-method for publicly known information-security vulnerabilities and exposures.}.

\bigskip
The aim of final report is to show the \textit{Prado2's} strengths and weaknesses, with the purpose of helping its administrators to make improvements the platform and, at the same time, helping them getting to know the opinion of the users.

\bigskip
As we know, the Internet has had an academic nature practically since its inception, which is why the greatest innovations and almost all of its structure emerged from this environment prior to the democratization of the web and global access to the Internet.  Nowadays it is unthinkable to conceive education without the use of the Internet, and in a not too distant future all teaching may be imparted through the Net. 

\bigskip
But a wide use is unconceivable without an appropriate, friendly, simple and even attractive user interface. This is why many companies spend millions of dollars improving theirs. ``Usability has come to stay''.\cite{jakonielsen} 

\bigskip
After the final implementation of the \textit{Prado2 platform} by the \textit{Rector’s Commission for a Digital University} (Delegación de la Rectora para la Universidad Digital), some students and teaching staff have showered disapproval on the platform’s deficiencies. The \textit{Granada University’s virtual education centre (CEVUG)} clearly needs more resources to provide a good service. This is the reason the powerful and well-established Moodle platform is  generating such criticism. In the words of \textit{Steve Krug} ‘If a system is unusable, no one will want to use it’\cite{stevekrug}.  

\bigskip
With respect to the initial analysis of the platform, we have found various errors that could be easily solved, greatly improving the ease of use.

\bigskip
The survey results were as expected and we have observed that the vast majority of users only need a platform for simple tasks as upload and download documents, and create tasks. We also observed that a good number of users would find it very useful to have better access from mobile platforms, either using a mobile browser adapted design or using a native smartphone app.

\bigskip
The analysis of the platform logs support the survey results. Some of the platform’s features is barely used, and this could be used to deactivate some options in order to simplify Prado’s usability.

\bigskip
In terms of security we demonstrate that there is much room for improvement and the platform should be amended as soon as possible, as a malicious user could steal and misuse information and documents from both teaching staff and from students.

\bigskip
I would like to highlight the great effort made by the \textit{CEVUG} professionals and I don't want this analysys to be understood as a critique towards their excellent job. With my experience as a web developer I can affirm that it’s not easy to keep in mind thousands of aspects required to customize, adapt and optimize a platform, more so when you work against the clock solving problems, which in many cases just allows making a ``temporary'' solution that stands in time indefinitely. This is why this report, although addressed to them as an aid to optimize the platform, maybe should make their superiors realize that something as important as the teaching support web platform should have more resources.

\bigskip
Just as an example, the Spanish National Police Corps uses a full time team of 8 people to manage their twitter account. The Prado platfrom, clearly more complex and requiring much more work actually doesn't have any developer and only counts with three part-time platform administrators. 


\newpage
\thispagestyle{empty}
\
\vspace{3cm}

\noindent\rule[-1ex]{\textwidth}{2pt}\\[4.5ex]

Yo, \textbf{\autor}, alumno de la titulación \textbf{\grado} de la \textbf{\escuela\ de la \universidad}, autorizo la ubicación de la siguiente copia de mi Trabajo Fin de Grado (\textit{\titulo}) en la biblioteca del centro para que pueda ser consultada por las personas que lo deseen.

\bigskip
Además, este mismo trabajo está publicado bajo la licencia \textbf{Creative Commons Attribution-ShareAlike 4.0} \cite{CC}, dando permiso para copiarlo y redistribuirlo en cualquier medio o formato, también de adaptarlo de la forma que se quiera, pero todo esto siempre y cuando se reconozca la autoría y se distribuya con la misma licencia que el trabajo original. El documento en formato {\tt LaTeX} se puede encontrar en el siguiente repositorio de {\tt GitHub}: \url{https://github.com/erseco/ugr_tfg}.

\vspace{4cm}

\noindent Fdo: \autor

\vspace{2cm}

\begin{flushright}
\ciudad, a \today
\end{flushright}

\newpage
\thispagestyle{empty}
\
\vspace{3cm}

\noindent\rule[-1ex]{\textwidth}{2pt}\\[4.5ex]

D. \textbf{\tutor}, profesor del \textbf{Departamento de Lenguajes y Sistemas Informáticos} de la \textbf{\universidad}.

\vspace{0.5cm}

\vspace{0.5cm}

\textbf{Informa:}

\vspace{0.5cm}

Que el presente trabajo, titulado \textit{\textbf{\titulo}}, ha sido realizado bajo su supervisión por \textbf{\autor}, y
autoriza la defensa de dicho trabajo ante el tribunal que corresponda.

\vspace{0.5cm}

Y para que conste, expide y firma el presente informe en \ciudad\ a \today.

\vspace{1cm}

\textbf{El tutor:}

\vspace{5cm}

\noindent \textbf{\tutor}

\chapter*{Agradecimientos}
\thispagestyle{empty}

\vspace{1cm}

A Georgia, que algún día será mejor ingeniera que su tío.

\bigskip
Al ZX Spectrum 128K de mis hermanos, porque sin él no habría llegado hasta aquí.
