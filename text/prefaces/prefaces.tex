\begin{center}
{\LARGE\bfseries\titulo}\\
\end{center}
\begin{center}
\autor\
\end{center}

\section*{Resumen}

\bigskip
\noindent{\textbf{Palabras clave}: \textit{\keywords}\\

Este proyecto tiene como objetivo la elaboración de un informe profesional sobre la \textit{Plataforma de Recursos de Apoyo a la docencia} en uso actualmente por la \textit{Universidad de Granada} analizando su {\tt Accesibilidad}, {\tt Usabilidad}, {\tt Seguridad} y Disponibilidad.

\bigskip
Para dicho análisis se realizaron encuestas a usuarios de la plataforma, se analizaron las estadísticas de uso en colaboración con los administradores de la misma y se analizó el código de la plataforma \textit{moodle}, se revisaron las recomendaciones de buenas practicas de sus desarrolladores y se estudiaron algunas de las vulnerabilidades documentadas en su CVE\footnote{Common Vulnerabilities and Exposures, siglas CVE, es una lista de información registrada sobre conocidas vulnerabilidades de seguridad, donde cada referencia tiene un número de identificación único.}.

\bigskip
El informe final tendrá como objetivo dar a conocer tanto las fortalezas como las debilidades de la plataforma Prado2 con objeto de servir de ayuda a sus administradores para mejorar la misma y ayudándoles a su vez a conocer la opinión de sus usuarios.


\newpage
\begin{center}
{\LARGE\bfseries\tituloEng}\\
\end{center}
\begin{center}
\autor\
\end{center}

\section*{Extended abstract}

\bigskip
\noindent{\textbf{Keywords}: \textit{\keywordsen}\\

This project’s objective is the development of a professional report on the Plataforma de Recursos de Apoyo a la Docencia (Teaching Support Resources Platform), currently in use by the Universidad de Granada, analyzing its accessibility, usability, safety and availability.

\bigskip
In order to develop this analysis, users of the platform have taken surveys, statistics of usage have been analyzed in cooperation with the administrators of this platform, the moodle platform has been analyzed, developer’s good practice recommendations have been inspected and some of the vulnerabilities documented in its CVE have been studied.

\bigskip
The final report’s objective will be showing the strengths and weaknesses of the Prado2 platform, with the purpose of aiding their administrators in the improvement of the platform and at the same time, helping them getting to know the opinion of the its users.

\bigskip
As we already know, Internet has had an academic nature practically since its birth, thus, the greatest innovations and almost all its approach has emerged from this circles previous to the net’s democratization and Internet’s global access.  It’s unthinkable nowadays conceiving education without the use of Internet, and maybe in a not too far away future, the whole of teaching will be done through the net.

\bigskip
But a wide use is unconceivable without an appropriate, friendly, simple and even attractive interface, this is why many companies spend millions of dollars improving theirs. ‘Usability has come to stay’.\cite{jakonielsen} 

\bigskip
After the final implementation of the Prado2 platform by the Digital University Rector’s Delegation (Delegación de la Rectora para la Universidad Digital) criticism has taken place among students and teachers towards the platform’s deficiencies. This, combined with the insufficient resources which the CEVUG (Virtual Teaching Centre of Granada University) needs to offer an adequate service, has precipitated that a potent and well established platform as moodle generates rejection, in the kind words of Steve Krug ‘If a system is unusable, no one will want to use it’\cite{stevekrug}.

\bigskip
With respect to the platform’s previous analysis, we found various errors that could be easily resolved improving in a great manner the easiness of usage.

\bigskip
The survey’s results were the expected and we observed that the vast majority of users just need a platform where they can upload and download documents, and create tasks. We also observed that a good number of users would find very useful a better access from mobile platforms, either using a mobile browser adapted design or using a native smartphone app.

\bigskip
The analysis of the platform’s registry back the survey’s results. A minimum percentage of the platform’s features is barely used, and this could be used to deactivate some options in order to make Prado’s usability easier.

\bigskip
In terms of security we demonstrate that it’s quite deficient and should be immediately improved, a malicious user could steal information and documents from teachers or students and make an illicit use of them.

\bigskip
I would like to stress the great effort performed by the CEVUG professionals and I do not wish that this analysis is understood as a criticism towards their excellent job. With my experience as a web developer I can affirm that it’s not easy to have in mind thousands of aspects required to customize, adapt and optimize a platform, more so when you work against the clock solving problems, which in many cases just allows making a “temporary” adjustment that stands in time indefinitely. This is why this report, although addressed to them as an aid to optimize the platform, maybe should make their superiors see that something as important as the teaching support web platform should have more resources.

\bigskip
Just as a referral, Spain’s National Police Corps uses a full time team of 8 to manage their twitter account. Prado, being clearly more complex and requiring much more work only counts with a full time developer and two part-time auxiliaries. 


\newpage
\thispagestyle{empty}
\
\vspace{3cm}

\noindent\rule[-1ex]{\textwidth}{2pt}\\[4.5ex]

Yo, \textbf{\autor}, alumno de la titulación \textbf{\grado} de la \textbf{\escuela\ de la \universidad}, autorizo la ubicación de la siguiente copia de mi Trabajo Fin de Grado (\textit{\titulo}) en la biblioteca del centro para que pueda ser consultada por las personas que lo deseen.

\bigskip
Además, este mismo trabajo está publicado bajo la licencia \textbf{Creative Commons Attribution-ShareAlike 4.0} \cite{CC}, dando permiso para copiarlo y redistribuirlo en cualquier medio o formato, también de adaptarlo de la forma que se quiera, pero todo esto siempre y cuando se reconozca la autoría y se distribuya con la misma licencia que el trabajo original. El documento en formato {\tt LaTeX} se puede encontrar en el siguiente repositorio de {\tt GitHub}: \url{https://github.com/erseco/ugr_tfg}.

\vspace{4cm}

\noindent Fdo: \autor

\vspace{2cm}

\begin{flushright}
\ciudad, a \today
\end{flushright}

\newpage
\thispagestyle{empty}
\
\vspace{3cm}

\noindent\rule[-1ex]{\textwidth}{2pt}\\[4.5ex]

D. \textbf{\tutor}, profesor del \textbf{Departamento de Lenguajes y Sistemas Informáticos} de la \textbf{\universidad}.

\vspace{0.5cm}

\vspace{0.5cm}

\textbf{Informa:}

\vspace{0.5cm}

Que el presente trabajo, titulado \textit{\textbf{\titulo}}, ha sido realizado bajo su supervisión por \textbf{\autor}, y
autoriza la defensa de dicho trabajo ante el tribunal que corresponda.

\vspace{0.5cm}

Y para que conste, expide y firma el presente informe en \ciudad\ a \today.

\vspace{1cm}

\textbf{El tutor:}

\vspace{5cm}

\noindent \textbf{\tutor}

\chapter*{Agradecimientos}
\thispagestyle{empty}

\vspace{1cm}

A Georgia, que algún día será mejor ingeniera que su tío.

\bigskip
Al ZX Spectrum 128K de mis hermanos, porque sin él no habría llegado hasta aquí.
