\chapter{Anexos}

\section{Respuestas a las preguntas de texto libre de la encuesta}

\texttt{\textbf{Aviso:} Para respetar la opinión de los participantes de la encuesta se han agregado todos los comentarios recibidos sin excepción. Queremos dejar claro que los mismos no reflejan nuestra opinión y que no nos responsabilizamos del contenido de los mismos.}

\subsection{¿Ha encontrado algún error que debería ser subsanado en \texttt{Prado2}?}

\begin{enumerate}
\item A veces intentas loguearte vía móvil y aunque pongas los datos correctamente te lleva de nuevo a la pantalla de inicio.
\item A veces no se encuentra disponible por mantenimiento. Se agradecería que avisaran con antelación de que no se va a poder acceder.
\item A veces se queda colgado completamente y me duplica los cursos, por ejemplo.
\item Absolutamente todo, PRADO2 es un sin sentido que vuelve loco a profesores y alumnos. Complicaciones a la hora de acceder a una asignatura, a pesar de que el profesor haya dado de alta.
\item Además si como en mi caso, se dió de baja de un grado, aparecen las asignaturas a pesar de la anulación de la matrícula.
\item Al poner una fecha de inicio y fin de una actividad, debería aparecer por defecto el año en que estamos. En una ocasión me ocurrió que la actividad se daba por cerrada un año antes.
\item Aplicación móvil. Mejor interfaz. Mayor velocidad
\item Básicamente todo, diseño, funcionalidad, cambios de grupo, etc, si lo queren poner de verdad que copien a swad, que es el que debería estar.
\item Cuándo te matriculas en un grupo de prácticas no te deja cambiarte de grupo en caso de que te hayas equivocado.
\item Debería ser más sencillo integrar los N grupos de teoría y prácticas en uno solo. Sobre todo cuando, como es mi caso, un solo profesor imparte todos los grupos de la asignatura. En estos casos, ¿qué sentido tiene que me vea con N copias de la asignatura en ``mis cursos''?
\item Demasiados.
\item Dependiendo del ordenador que uso se ve la pantalla principal de acceso, como si estuviera mal codificada. En cuanto los mensajes que recibo, entro para leerlos y sigo teniendo el aviso de mensaje nuevo cada vez que entro. Por otro lado, aunque se que no es error de \texttt{Prado2}, siguen apareciendo asignaturas de otros años, cuando ya han sido cursadas, esto incomoda, sería mas funcional que solo salieran las asignaturas que se están realizando en el momento.
\item Desde Chrome en mi dispositivo movil no puedo acceder a las asignaturas. Al hacer login me vuelve a la pagina de inicio una y otra vez para volver a loguearme. Desde otros navegadores sí puedo. Desde el móvil es muy complicado pulsar en el icono de ajustes (las tres barritas azules) por lo que ver las calificaciones es casi imposible porque desaparece al hacer scroll.
\item Desde el navegador de mi móvil no puedo acceder, ya que me al loguearme me redirecciona a la página principal, como si no hubiera accedido, y nunca puedo comprobar calificaciones
\item Desde mi móvil samsung con el navegador no puedo entrar a \texttt{Prado2}. Introduzco la contraseña entra y cuando le pulso a cualquier categoría o apartado se desconecta mi sesion
\item Dificil cambiar entre las diferentes asignaturas y una vez que hay muchas es dificil manejarse entre ellas, ademas de que en algunas asignaturas existen varios grupos(comun, profesor en concreto, profesor de practicas...). Tambien deberia mejorar su interfaz en dispositivos moviles ya que ahi si que es costoso cambiar entre asignaturas o ver las calificaciones, ...
\item Dificil de manejar, debería tener una opción en la que el alumno pudiera escribir datos personales que el profesor pidiera de forma libre
\item El inicio de sesión desde Android puede volverse tedioso e interrumpirse numerosas veces
\item El problema de la adjudicación de grupos de alumnos en el máster de educación. Aparecen los alumnos de todas las especialidades en la misma asignatura y todos los profesores pueden subir o borrar recursos de la plataforma, aunque no sean profesores de ese grupo en concreto.
\item El sistema de mensajeria dentro de \texttt{Prado2} es muy lioso, y al escrubir mensajes los profesores no saben responder y luego no se encuentran con facilidad donde puedes encontrarlo.
\item El tema de las calificaciones de las tareas on line. Las tareas valen por ejemplo 0.5 puntos, y no se encuentra esa puntuación
\item El tiempo de.espera para publicar en los foros
\item El usuario y contraseña no se queda registrado. Hay que conectarse cada vez que se usa.
\item En general la usabilidad es muy mala
\item En general los profesores no saben utilizar la opción de elegir grupo, optando por pasar una hoja en papel y que cada uno se apunte donde quiera.
\item En una ocasión probé a poner una práctica en la que había que usar el chat, pero los alumnos me dijeron que era muy poco funcional. Al parecer, costaba seguir una conversación, resultan preferibles los foros.
\item Es imposible acceder a los mensajes una vez leídos.
\item Es incomprensible
\item Es lento y en el fin de semana se solía caer el sistema y la entrega de prácticas en esos días era un caos
\item Es muy difícil acceder y usar a los foros, además, las novedades no aparecen todas en un tablón conjunto y los profesores no tienen forma de saber cuándo nos llegan o no los mensajes que nos dejan.
\item Es muy lento. Es incómodo el proceso de añadir información (scroll en la pantalla)
\item Es muy pesado para personas que tenemos que acceder a los archivos a diario la cantidad de menús por los que hay que pasar y lo poco visuales que son (listas con enlaces).
\item Es muy poco intuitivo, es muy difícil encontrar cómo utilizar un recurso
\item Es poco intuitivo. La carga de las páginas, al guardar y modificar un contenido, induce a pensar que uno se ha equivocado. La carga suele ser lenta y anárquica.
\item Es una locura el tema de la matrícula de los estudiantes; me han escrito estudiantes para decirme que a pesar de estar en otro curso y no haberse matriculado nunca en mi asignatura, aparecen y, por tanto, les llegan los mensajes y tienen acceso a la plataforma. Por el contrario, hay muchos estudiantes que tengo que matricular manualmente porque, a pesar de estar en mi grupo, no aparecen!!!
\item Escribir a los compañeros de mi clase y que llegue a ellos y no a todo el mundo
\item Espero que eliminen Prado el próximo curso.
\item Existen muchos errores en la herramienta, principalmente de usabilidad. Empezando por la parte en la que se pone texto blanco sobre fondo blando, a los menús sin AJAX, etc etc
\item Falla al cargar el CSS, se cierra la sesión con Google Chrome en Android
\item Fallos en la versión para móvil, por ejemplo, no se puede acceder a las calificaciones.
\item Hacer la búsqueda de mensajes más fácil
\item Hay algunas funciones inaccesibles o difíciles de encontrar en dispositivos móviles.
\item Hay muchos errores más pero ese me parece importante
\item Hay profesores que no saben como hacer publicos los archivos
\item Incorporación en asignaturas/grupos de cursos pasados o en los que no estoy matriculado
\item La barra de menú en el móvil no funciona
\item La organización de los archivos y temas es muy liosa. Que se eliminen las asignaturas que ya has cursado en otros cuatrimestres.
\item La plataforma en general carece de estilo y orden alguno, las cosas están por medio sin una estructura de calidad.
\item La plataforma funciona con lentitud y las caídas son muy frecuentes, en ocasiones en horas críticas para la entrega de ejercicios (entre las 22:00 y las 00:00)
\item La página se cae demasiado.
\item La usabilidad deja mucho que desear, la plataforma no es nada intuitiva.
\item Las herramientas de grupos/agrupamientos me han dado algo de trabajo. Creo que ya lo controlo, pero me da la sensación de que podrían ser más fáciles.
\item Las sesiones de usuario (sobre todo cuando se visualiza desde el smartphone) se mantienen de una forma muy rara, ha habido veces que se ha cerrado sin motivo alguno y hay que volver a iniciar sesión. Adicionalmente, se debería habilitar un botón para recordar la sesión.
\item Llegan mensajes del TFG a todos los grados y cursos.
\item Los correos por parte de los profesores se envían a toda la lista de alumnos de esa asignatura cuando, dicha asignatura, depende de varios profesores y no se tiene clase con todos ellos.
\item Los errores que he encontrado los he comunicado al CEVUG y los han resuelto rápidamente.
\item Los mayores problemas que creo que se deberian subsanar son todos de usabilidad. No es para nada intuitivo.
\item Los mensajes de los profesores desaparecen una vez abiertos
\item Los profesores suben archivos y no se nos notifica a los alumnos
\item Menu global en version movil dificil de acceder
\item Muchas veces la página se queda colgada
\item Muchos
\item Muy desorganizada la plataforma
\item Muy necesario que al acceder a la plataforma, nos aparezcan las nuevas notificaciones. No que ahora tenemos que estar accediendo asignatura por asignatura revisando si hay novedades y documentos y mensajes nuevos enviados por profesores.
\item Más agilidad en los mensajes
\item NO
\item Nada intuitivo
\item No
\item No avisar de la subida de nuevos archivos
\item No deja cambiarte de grupo una vez que entras en uno
\item No hay cojones a eliminar asignaturas. El orden general de las asignaturas es lioso. Se reciben mensajes de asignaturas en las que no estás (ni has estado nunca) matriculado. Es complicadísimo mandar mensajes.
\item No hay notificaciones claras en la página principal de las novedades.
\item No poder incluir profesores en los cursos en calidad de profesor
\item No puedo acceder desde el móvil
\item No se borran las asignaturas ya aprobadas como ocurría en el tablón de docencia.
\item No se borran los cursos de las asignaturas una vez finalizado su uso trimestral o anual.
\item No se muestra con claridad los archivos nuevos o novedades en general, de forma que el usuario tiene que estar buscando qué ha visto ya y qué no.
\item No se puede acceder correctamente desde Chrome en dispositivos móviles.
\item No sé de quién depende esto pero, por favor, que aparezcan en moodle todos los estudiantes matriculados en la asignatura, y sólo ellos.
\item No sé.
\item No.
\item Notificaciones
\item Parece un sabotaje del trabajo del profesor, supongo que por falta de personal
\item Por ahora no
\item \texttt{Prado2} es un error en sí
\item Privacidad
\item Que los alumnos puedan borrar los cursos que ya no necesitan
\item Que no avisa a los profesores de que su asignatura no está visible a los alumnos
\item Que no exista como aplicación para móvil y que a veces no me deje entrar.
\item Que no se borran las asignaturas que ya no estas matriculado
\item Quitar automáticamente alumnos que se han dado de baja de la asignatura
\item SE BLOQUEA EN OCASIONES
\item Se cae con cierta frecuencia.
\item Se cae constantemente
\item Si
\item Si envío un mensaje a un estudiante a través de moodle, no puedo poner un 'Asunto'. Imagino que simplemente le sale en 'Asunto' el nombre de la asignatura, pero me gustaría poder poner algo más específico.
\item Si hago colapsar las columnas de navegación laterales (flechas izquierda y derecha), en cuanto pincho en otro enlace me vuelven a aparecer. Tengo que estar colapsándolas todo el rato y es un incordio.
\item Si, me siguen apareciendo asignaturas en las cuales no estoy matriculado
\item Sí, a los dobles grados se nos duplican las asignaturas, de forma que al final tenemos la misma asignatura duplicada, haciendo de \texttt{Prado2} una extensa llanura.
\item Sí. Al acceder desde Android (en una tablet) me cerraba sesión al elegir asignatura o justo después de haberla iniciado.
\item Sí. Sólo nos llegan notificaciones al correo de las novedades, pero muchos profesores cuelgan temario sin poner en novedades nada y no hay forma de saber si hay alguna novedad salvo que se mire detenidamente
\item TODO
\item Tantos que no caben en este espacio
\item Tendría que ser mucho más accesible en todos los aspectos, solo para iniciar sesión tienes que buscar donde hacerlo.
\item Todos los cursos pongo un cuestionario de valoración de la asignatura que los estudiantes pueden responder anónimamente. moodle preserva el anonimato en las respuestas, pero no en el 'registro de actividad'. Es decir, se puede saber quiénes respondieron al cuestionario e incluso, si uno estuviera muy pendiente de cada actualización de respuestas, quién es responsable de cada respuesta. Creo que cuando una actividad se etiqueta como anónima, el acceso a la misma debería desaparecer del registro de actividad.
\item Usa un lenguaje casi críptico, poco natural, es difícil de entender qué significa cada cosa.
\item Varios, ya los reporte y se corrigieron.
\item demasiado complicado
\item la accesibilidad falla y hay que repetir el proceso de acceso
\item la adaptabilidad a dispositivos moviles, vision y descarga de pdfs desde la version movil
\item no
\item ¿Uno solo?
\end{enumerate}


\subsection{Indique si hay alguna función adicional que le gustaría ver en \texttt{Prado2}}
\begin{enumerate}
\item (1) Autoselección de grupo: poder tener un número diferente de alumnos en cada grupo. (2) Cuando hay alteraciones de matrícula, debería informarse en PRADO (por ejemplo, si hay alumnos que ya no están matriculados, debería aparecer un mensaje al logearse para que de forma cómoda se puedan eliminar de \texttt{Prado2} o mantenerlos durante un tiempo). Creación automática de grupos/agrupamientos para distintas convocatorias
\item App para móviles
\item Cada día es más necesaria una aplicación para Android e iOS para la visualización de la sesión sin necesidad de entrar a internet e iniciar sesión cada vez (para empezar se podría implementar un simple WebView, y poco a poco ir haciéndola más nativa de cada plataforma).
\item Chat grupal.por clases
\item Creo que debería haber algún tipo de app, que mande notificación cuando los profesores suban algún archivo o calificación
\item Creo que debería haber algún tipo de app, que mande notificación cuando los profesores suban algún archivo o calificación
\item Debería habilitarse una entrada para entregas de tareas con turnitin
\item Demasiadas.
\item Descargar todo el material de las asignaturas con un solo botón.
\item Desconocen el concepto de user friendly
\item Diseño móvil que funcione correctamente.
\item El diseño en general es mejorable. Es muy dificil encontrar las cosas.
\item El problema de una asignatura, ligada a muchos grupos o cambios de estudio. Lo mejor una sola lista y no multitudes.
\item Eliminar asignaturas, maletín, notificaciones cuando se suben nuevos ficheros y un largo etcétera.
\item Eliminar una asignatura ya cursada desde el perfil estudiante
\item En el antiguo tablón de docencia aparecía una especie de reloj cuando había alguna novedad, me gustaría ver en Prado algo similar.
\item Estabilidad y que cuando se suban las notas, suban cosas importantes como un comentario en una practica previamente entregada o una notificacion que por ejemplo el profesor no puede asistir a clase, que se notifiquen al correo ugr
\item Evaluación profesorado
\item Fotografías de alumnos
\item Función antiplagio con en Turnitun o Ephorus
\item HACERLO SENCILLO Y PRÁCTICO PARA USUARIOS NO EXPERTOS
\item Incluir el Correo UGR dentro de \texttt{Prado2}
\item Integración con Jupyter Notebooks
\item Interfaz responsive en móvil
\item La de reprogramarlo
\item La desmatriculación de asignaturas, una vez superada o pasado el año.
\item La función de creación de Web Quest
\item La introducción de las calificaciones lleva muchísimo tiempo.
\item Las asignaturas que hemos cruzado o caambiado de magricula no desaparecen sino que se genera una lista muy larga de asignaturas donde es bastante dificil encontrar la que necesitas en el momenco concreto. Seria útil si pudieramos o quitar o mas bien ocultas las asignaturas las que no nos hace falta. Asimismo, espablecer el orden en el que aparecerian o deliberamente, o automatico segun la frecuencia en la que usamos la asignatura concreta. Gracias por oportunidad de expresar nuestra opinion y espero que las sujerencias razonables y logicas se tendran en cuenta.
\item Libreria temática
\item Los pos-it de avisos (urgentes), como en swad
\item ME GUSTARÍA LA FUNCIÓN DESTROY PLATAFORM
\item Mayor personalización del panel que muestra las asignaturas, poder borrar u ocultar de la lista las asignaturas que en un determinado momento no interesen.
\item Me encantaría entrar un día en Prado y ver que no existe, que es el antiguo SWAD o tablón de docencia.
\item Me gustaría que no apareciesen las asignaturas ya superadas
\item Me gustaría ver menos cosas y mas útiles y accesibles. Llevo 4 años usando Prado y aun no se acceder a la mayoria de las cosas del jaleo que hay montado. Mas simple como DECSAI sería mejor.
\item Me gustaría, que cuando suban los profesores documentos a la plataforma, en cada una de las asignaturas salga una alarma en el título de la asignatura (menú principal), y luego al pinchar en ella, otra alarma en el documento, actividad o tema donde haya habido cambios nuevos, como pasaba en el tablón de docencia. Es muy molesto tener que meterte 3 veces al día en asiganatura por asignatutura y pestañitas de temas o actividades, uno por uno, para ver si alguien ha subido algo nuevo; les quita mucho tiempo a los estudiantes.
\item Mejor ordenacion de asignaturas, tal vez por año o por tipo en algo parecido a carpetas o apartados. La mensajeria podria mejorar muchisimo todavia. Swad ya implementa todas estas cosas de una forma mucho mas comoda, arriba cañas
\item Mejor usabilidad con respecto al mapa web del sitio. Confusión cuando se va de una carpeta a otra.
\item Más que función , que funcione
\item Módulos de corrección de programación tipo CodeRunner
\item Nada
\item Necesita un reforma en general
\item Ninguna
\item No
\item No notifica cuando hay nuevos archivos subidos por los profesores. Debería tener un sistema de notificaciones para estos casos.
\item No puedo responder a esta pregunta seriamente.
\item Notificaciones (documentos/mensajes no vistos)
\item Notificaciones de subida de documentos por parte de profesores, y notificaciones por calificaciones.
\item Notificaciones en tiempo real y eliminarme de asignaturas ya aprobadas
\item Notificaciones, indicacion de material nuevo y modificado, mejor acceso a mensajes
\item Notificación al correo sobre las novedades
\item Notificación de documentos subidos por profesores
\item Notificación de las nuevas publicaciones en \texttt{Prado2}
\item Personalización de los grupos en los posgrados y practicum.
\item Poder descargar todos los archivos a la vez
\item Poder eliminar asignaturas ya cursadas. Cada vez que se accede a ``mis cursos'', aparecen todas las matriculadas de otros años
\item Poder migrar asignaturas a SWAD
\item Poder subir mas de un archivo a la tarea
\item Por asignatura debería haber solamente una carpeta.
\item Que anunciase si se han subido documentos nuevos como hacía SWAD
\item Que cada grupo tuviera una asignatura sin tener que hacer subgrupos ni nada por el estilo. Como estaba en el tablón de docencia por ejemplo.
\item Que las asocisciones puedan añadir cursos o actividades que ellos realicen y los estudiantes puedan acceder
\item Que los alumnos puedan darse de baja de las asignaturas que ya no tengan.
\item Que no se cierre cada X, y que sea más fácil entrar. Que sea más accesible desde móviles o que haya una aplicación.
\item Que salga una notificacion al lado de la asignatura que tenga nuevo contenido.
\item Que se pudieran ver en la página principal todas las asignaturas que se estan cursando
\item Que se viera bien en el móvil, que mande notificaciones con cada vez que el profesor suba algo o lo deje ver,.....
\item Que sea amigable
\item Que sea swad
\item TUTORIAS ON LINE
\item Test de la asignatura. (Batería de preguntas/ejercicios)
\item Tiene un formato poco práctico, demasiado cargado. Me gustaría que fuese más simple.
\item Un aviso de los nuevos ficheros, Ordenar las asignaturas en la portada.
\item Un resumen de calificaciones detallado, ademas de un calendario que se mas facil de acceder
\item Una aplicación móvil
\item Una configuración de la interfaz mayor por parte del usuario, un calendario donde aparezcan la fecha de entrega de trabajos y/o exámenes.
\item Una notificación cuando haya alguna novedad, como había en el tablón de docencia.
\item Una nuevo sistema fácil de usar y ayude a los profesores y estudiantes.
\item Una opción para descargar en formato .zip, en lugar de ir descargando cada archivo individualmente
\item Una pestaña de Notificaciones
\item Uso con frecuencia las bandejas de entrega para los trabajos de los estudiantes. Para descargarlos en mi ordenador, tengo que ir recorriéndolos uno por uno y pinchando en cada documento. ¿No podría habilitarse una función del tipo 'Descargar todo' para bajarse los documentos de esa bandeja de una sola vez?
\item Version movil
\item Ya que se envían al correo la participación en los foros, también sería util que se enviara una notificación cada vez que el profesor/a cuelga algo nuevo.
\item Zona de notificaciones generales
\item chat directo,
\item integrar los N grupos de teoría y prácticas en uno solo ;)
\item me gustaría que las que tiene funcionasen bien.
\item ninguna
\item no
\item notificaciones de temario añadido
\item poder dar de alta non-editing teachers
\item poder descargar carpetas
\item poder seleccionar el grupo de clase
\item quitar asignaturas ya cursadas
\item ver los mensajes como si un gestor de correo electrónico se tratase y el botón de acceso más accesible
\item volver a tablon de docencia
\item ¿Hasta qué punto los alumnxs pueden subir materiales para compartir con el resto de la clase por PRADO2 y no hacerlo a través del profesorado?
\end{enumerate}

\subsection{Indique cuál es la funcionalidad que más le gusta de \texttt{Prado2}}

\begin{enumerate}
\item Actividades
\item Al menos los cuestinarios y entregas que son de trabajo sobre la pagina web ejemplo, fundamentos del software se guardan los cambios automáticamente
\item aviso de mensajes por email
\item Calificaciones
\item Cantidad de archivis
\item Centralización de todas las asignaturas.
\item Clasificación temas
\item Comunicación bidireccional con el alumno, confidencial y versátil
\item Comunicación on line con el alumnado y registro de actividades
\item Cuestionarios autoevaluables
\item cuestionarios y tareas
\item Depende, en el sentido de dentro de lo malo que me gusta mas, se puede decir que dentro de una asignatura no esta mal la ordenacion si se lleva a cabo bien. Sin embargo hay que decir que parece ser(tampoco lo se con certeza) que parte del profesorado todavia no sabe usar bien las herramientas que ofrece \texttt{Prado2}, he vivido muchos casos en los que el profesor/a despues de un par de clases con complicaciones en el control de asistencia han renunciado y han recurrido al papel y boli lo cual es vergonzoso no para el profesorado si no para \texttt{Prado2}
\item El almacenamiento de lecciones.
\item El calendario aunque faltan detalles que pulir
\item El envio directo al correo de notificaciones cuando hay algun cambio
\item El listado de asignaturas
\item El orden de la plataforma
\item El profesor puede avisar a toda la clase con un sólo mensaje.
\item El que tenga un foro interno.
\item El sistema de evaluación online
\item Entrega de documentos.
\item Entrega de tareas y poder calificarlas y que el alumno puedo visualizarlas individualmente
\item Es sencillo, cómodo y muy intuitivo
\item Facilidad de edición y flexibilidad de organización de contenidos
\item Foro
\item Foros
\item He renunciado a usarla. A lo mejor necesita manuales más grandes que Hamlet
\item La accesibilidad a los documentos facilitados por los profesores
\item La capacidad para hacer cuestionarios de las asignaturas
\item La clasificación del temario.
\item La clasificación por temas en las asignaturas, asistencia y calificaciones
\item La compartición de material y las tareas
\item La copia de seguridad integral de un curso (y la restauración en otro momento).
\item la encuensta
\item La entrega de tareas con comentarios
\item La facilidad para entregar las tareas
\item La forma de las calificaciones y adjuntar archivos.
\item La forma en la que se hacen las entregas de trabajos
\item La gestión de calificaciones
\item La limpieza
\item La llegada de los mensajes de los profesores
\item La opción GA3 es bastante innovador, aunque no sé si por derechos de autor o no, no se permite la descarga del vídeo para su visualización offline.
\item La organización de las calificaciones.
\item La posibilidades de edición de la información que da HTML
\item La tarea y los cuestionarios son las funcionalidades que más me gustan.
\item La visibilidad de la plataforma, los temas, tareas, etc.
\item Las carpetas y archivos. Me basto con ellos perfectamente para poner cualquier archivo que necesito: audios, libros, presentaciones, etc. Las funciones específicas, en ese sentido, a mí no me resultan útiles, pero eso es algo personal
\item Las entregas de ejercicios funciona bien. Es lo mejor que tiene.
\item Las notificaciones al correo ugr cuando se añade un archivo
\item Las novedades, porque te llegan al correo institucional
\item Los cuestionarios
\item Los foros me resultan muy útiles.
\item Los foros y las bandejas de entrega son sin duda las más prácticas.
\item Mayor organización dentro de las asignaturas al clasificarse por ejemplo por temas.
\item Nada
\item ninguna
\item Ninguna es mejor que las de otras plataformas como SWAD.
\item Ninguna, porque encima de todo se cae para dos por 3
\item Ninguna.
\item no
\item no hay ninguna especial, dentro de las utilidades que uso
\item No hay tal funcionalidad. \texttt{Prado2} es inútil.
\item no lo sé
\item No me aporta más de lo que podía hacer en SWAD
\item No me gusta la plataforma en sí
\item No me gusta.
\item No soy capaz de ver funcionalidad en la plataforma al ser tan poco usable.
\item NS/NC
\item Poder acceder a los contenidos independientemente de la hora.
\item poder acceder a todas las asignaturas desde un boton
\item Poder acceder al material de clase.
\item Poder colgar materiales, tareas, recibir los trabajos y el foro
\item Poder entregar las tareas a través de dicha plataforma así como llevar un control del curso(calificaciones, asistencia, trabajos, etc.).
\item Poder enviar prácticas fácilmente y saber cuál es el plazo para entregarlas.
\item Podría ser el de calificaciones y subida de ejercicios o trabajos pero ningún profesor hace uso de ello.
\item Posibilidan de acceso al material ofrecido por el profesor y autoevaluaciones.
\item \texttt{Prado2} no tiene funcionalidades fuera de lo común con respecto a otras plataformas.
\item Pruebas multirespuesta randomizadas
\item Pruebas virtuales
\item Que dentro de la asignatura el docente puede clasificar los temas por carpetas. Es más fácil de ver todo.
\item Que puedes ponerte en contacto con los profesores de las asignaturas y puedes obtener su correo institucional
\item Que se accede con el correo de la facultad, por poner algo.
\item Que te avisa a veces por correo cuando suben algo.
\item Revisar una entrega con el profesor directamente, pero no lo utiliza nadie.
\item Se pueden subir temas organizados en bloques dentro de cada asignatura.
\item Sección de calificaciones, es clara
\item Su organización
\item Su permanencia. Ya está claro que va a quedarse hasta el fin de los tiempos.
\item Su seguridad y el aviso de mensajes nuevos.
\item subir los temas
\item Subir y bajar archivos y comunicar.
\item Tareas
\item versatil
\end{enumerate}

\subsection{Comentarios adicionales}

\begin{enumerate}
\item A mi me gusta más que las otras plataformas que he usado, es más sencilla y completa en cierto modo.
\item A mi me parece que la plataforma es muy útil pero su uso, sometido a actualizaciones constantes (lógicas y necesarias), se hace a veces insufrible. Entiendo que falta soporte, no necesariamente por medio de persona, sobre todo cuando se cambia de versión y el entorno es radicalmente nuevo. Esta necesidad de apoyo se hace perentoria cuando se tiene en cuenta lo poco intuitivo que es el uso de moodle en muchísimas situaciones. Así se consigue que el uso de moodle no sea todo lo rico que debiera ser. La formación que se da es para mínimos y normalmente se plantean problemas que exceden esa formación.
\item A pesar de la crítica anterior (los N grupos por asignatura), aprecio el trabajo para intentar satisfacer tantos y tantos requisitos de tantos y tantos profesores ¡tan peculiares! ;)
\item Ademas de prado, como ya he mencionado, tambienhe probado swad y decsai. He de decir que ambas me parecen mucho mas intuitivas y faciles asi como rapidas. Tanto para el profesorado como para el alumnado. Diria que swad ahora mismo es mucho mas potente que prado en muchos sentidos (si esque tiene hasta una especie de twitter, viva cañas de nuevo) e incluso con la ``simpleza'' que alberga decsai, el cual simplemente tiene entregas, asistencia, archivos y calificaciones (dicho un poco en bruto) es mucho mas comodo de usar, no entiendo el por que de intentar unificar todo en prado si ni siquiera es capaz de llevar a muchos usuarios a la vez ( ha habido muchos dias en los que iba muy lento) mientras que el twitter de swad se dedica a subir estadisticos de lo chulo que esta ver el pico de actividad cuando se dan notas o situaciones parecidas sin problema ninguno
\item Algunas asignaturas me aparecen hasta tres veces, y el profesor cuelga archivos indistintamente en un grupo o en otro sin distinguir en qué grupo estoy matriculado.
\item Algunas opciones de uso común tienen varios pasos intermedios que creo innecesarios
\item Algunos de los documentos que suben los profesores no aparecen. En general no es ``amigable"
\item Bastante mala en cuanto a diseño. Muy mejorable el apartado de visualización en smartphones.
\item Caótico y desastroso sin duda.
\item Creo que aún le faltan muchas modificaciones para llegar a ser una plataforma de fácil uso y más util que las que había anteriormente.
\item Creo que ha sido un paso atrás respecto a SWAD. Esta misma tenia aplicación móvil desde hace años. Aparte de que era mucho mas intuitiva.El diseño de la pagina es horrible, y al principio resulta complicado de entender, todo lo contrario a otras plataformas como eran el tablón de docencia y swad.
\item Creo que no han acertado con esta plataforma
\item Creo que PRADO necesita: (1) interfaz usable, la que tiene es de lo peor que he visto (continuamente plegando/desplegando menús, texto/espacios muy grandes lo que implica scrolling continuamente, hace tiempo que no lo uso desde móvil y no sé si se ha arreglado pero antes funcionaba muy mal, configuración por defecto más funcional ... ahora aparecen un montón de bloques que no se usan asíduamente) (2) una capa que haga usable el sistema de calificaciones (poner notas a trabajos, calcular la nota final, etc), lo que hay ahora es muy potente pero es una tortura usarlo. Creo que un porcentaje muy alto del PDI usa sistemas de evaluación similares por lo que podría simplificarse (3) más recursos, a veces es extremadamente lento (4) que las tareas de mantenimiento no se hagan en horario lectivo, alguna vez he tenido previsto un examen y ¡sorpresa! no estaba operativo ... (5) Al comenzar nuevo curso debería automatizarse al máximo la creación del curso en base al curso anterior (6) La disponibilidad de los cursos debería ser inmediatemente posterior a la aprobación del POD y no pocos días antes de comenzar el curso (no sé qué plazos hay ahora).
\item Creo que sería buena idea que una vez que se finalice una asignatura, se elimine automáticamente o te deje desinscribirte de forma manual para poder acceder a los cursos de interés más fácil y rápidamente, ya que si se amontonan los cursos, algunos de ellos se ''ocultan'' y se tarda ligeramente más en poder acceder a cursos que necesitas consultar más periódicamente y otros cursos que no utilizas se encuentran más accesibles.
\item Creo que todavía le falta desarrollo, porque para mi parecer faltan muchas cuestiones que deben de quedar más claras para poder utilizarla con una mayor eficacia.
\item Debería existir una opción para que los alumnos pudieran quitarse de grupos de asignaturas y no fueran los profesores los que lo hicieran.
\item Debería ser más simple. Tareas teoría y sus fechas de entregas, tareas prácticas y sus fechas de entregas, apartado de teoría, calificaciones, correo interno. Echo en falta que el estudiante pueda borrarse de alguna asignatura en la que aparece matriculado ya que si el profe al empezar nuevo curso no nos borra la lista de asignaturas matriculadas es horrible! Y más ahora que tenemos las asignaturas repetidas por profe de teoría y por profe de Prácticas. Es un rollo esta plataforma no es nada intuitiva y con tanta información que maneja cuesta mucho encontrar las cosas.
\item Deberían eliminarla, o por lo menos no intentar implementarla como plataforma digital obligatoria para todas las carreras, ya que por ejemplo SWAD aunque parezca más rudimentaria en cuanto a apariencia, su manejo y accesibilidad son muy sencillos, facilitando mucho el trabajo.
\item Desde que está \texttt{Prado2} sólo hace que entorpecer el funcionamiento adecuado de descarga de apuntes y subida de trabajos. Últimamente está casi siempre caída, funciona lenta, no es intuitiva, tiene muchas cosas inservibles... Swad o decsai son mucho más intuitivas y simples para cualquier persona de la comunidad universitaria y mucho más eficientes.
\item devolvernos tablón de docencia
\item Echo de menos el tablón de docencia por su sencillez, a pesar de su aspecto rudimentario. \texttt{Prado2} es muy poco práctico, muy desordenado, tiene menús inservibles por todos lados, incómodo de navegar, poco intuitivo, y para colmo aparecen muchísimas asignaturas de las que no estoy matriculado y de las cuales también recibo notificaciones por correo electrónico. Envié un reporte a \texttt{Prado2} y me contestaron que era el profesor quien me tenía que dar de baja de las asignaturas que no estuviera matriculado. Pero los profesores o no saben hacer eso o no quieren hacerlo porque les dé igual.
\item El estado actual de la plataforma yo no lo veo válido como para estar ahora mismo en producción y que sea la plataforma oficial de la UGR. Casi parece estar más en una fase temprana (muy temprana) de desarrollo que un proyecto acabado. La interfaz es muy poco intuitiva, la versión web para móviles es muy muy mala, etc. En definitiva, hay muchas cosas que mejorar y mucho camino por recorrer. Creo que ya hay otras alternativas como plataformas de la UGR que funcionan mucho mejor en términos de usabilidad, disponibilidad, facilidades de uso, etc.
\item El profesorado NO sabe usar \texttt{Prado2} en la inmensa mayoría de los casos
\item El rendimiento y la robustez de prado es ridícula.
\item El sistema de acceso es lento... primero, pantalla de Prado; luego, salto al sistema de identificación; otro salto más a la página principal de Prado; luego otra más para acceder a los cursos... Si se pudiera acceder a la identificación desde el propio Prado se ahorrarían una cantidad de clics importante.
\item Eliminen \texttt{Prado2} y dejen Swad, la estupidez humana demuestra que es infinita con este tipo de cambios que no sirven de nada y hace que todo vaya un poco peor.
\item en comparacion con el tablon de docencia este medio lo veo obsoleto, lento y falto de criterio...sigo apostando por el Tablon de Docencia mas agil y dinamico para nosotros los estudiantes
\item En menos de la mitad de las asignaturas los profesores han sido capaces de darnos acceso a la información a la primera, entiendo que los profesores deben preocuparse de aprender a manejar la web pero quizás se debería hacer el uso un poco más sencillo para ellos, ya que los que sufrimos esos problemas, al final, somos los alumnos.
\item En mi opinión, es un poco complicada por el tema de no tener notificaciones visibles en la pantalla principal, como si se veían en la plataforma Swad. Por lo demás no me parece una mala plataforma.
\item Es bastante horrible, nada intuitivo y hay cosas que no llego a ver después de dar mil vueltas. Ves muchas cosas innecesarias y no ves las necesarias. Echo mucho de menos el tablón de docencia que, con sus carencias, al menos resultaba más simple e intuitivo. El SWAD está mucho mejor diseñado.
\item es mala, cumple funciones, pero intermitentemente, con errores, muy poco intuitiva
\item Es necesario seguir formando a estudiantes y profesores en todas las funciones posibles de prad2
\item Es una plataforma muy potente, pero el tiempo de enseñanza es corto y no tenemos tiempo (por mi parte) para utilizar ese potencial; sobre todo si damos clases presenciales. Se trata de doble trabajo, lo que conduce a una reducción del uso de la plataforma.
\item Es una plataforma no apta para usuarios normales.
\item Espero que se cree una nueva web el año que viene. Si no es así usaré Facebook. Cuesta mucho comprender que hayan quitado el Tablón de docencia para hacer esto. Gracias.
\item Esta encuesta se podria haber realizado usando el plugin SEPUG que permite rellenarla por movil y desde prado. Además como encuesta general dispondría de mas visibilidad
\item Estaria bien una aplicacion movil
\item Estaría bien que todos los profesores usaran Prado, pero algunos se resisten. ¿No pueden hacerlo obligatorio?
\item Está por resolver dar respuesta al funcionamiento de algunas asignaturas, donde la casuística de grupos de cabecera, grupos de prácticas y profesorado y alumnado que trabajan en cada uno de ellos es muy diversa, no siendo suficiente la propuesta actual de asignatura común y asignatura particular.
\item Hay que hacer dos accesos lo cual ralentiza las gestiones.
\item La docencia, al menos en el tablón de docencia, era mucho más intuitiva que en prado 2; aquí tienes que pasar un rato dando vueltas por todas las pestañas para conseguir ubicar algo, porque en ningún momento te queda claro dónde estás tú o dónde está los documentos, o si estás o no dado de alta en una asignatura.
\item La forma de acceder a una asignatura tienes que leer toda la lista, deberían de replantear la visualización de la misma ,es decir un nuevo diseño intuitivo y más ágil y rápido
\item La herramienta ``Mensajes'' creo que es mejorable, y la designación-identificación de las asignaturas también. Enhorabuena de cualquier caso.
\item La idea de la implementación de \texttt{Prado2} me pareció muy buena (como usuario de moodle en apoyo a la docencia desde 2010 la estaba esperando). La idea de su obligatoriedad, al aparecer en sustitución de Swad y Tablón de Docencia, no me parece tan buena, habida cuenta la escasa buena acogida que aprecio en mi entorno de trabajo.
\item La información de apoyo, ayudas y tutoriales a las personas usuarias de la plataforma es escasa y hostil (p.e., yo me he enterado en mar-17 que existen formularios para peticiones al Cevug en relación con la plataforma).
\item La mayoria de profesores no conoce su funcionamiento: no saben como darte de alta/baja en sus asignaturas ni como darte acceso a los alumnos a los archivos que ellos mismos suben a la plataforma. Deberian de realizar algun curso previo antes de implantar una nueva plataforma de uso diario en UGR
\item La organización es pésima
\item La peor plataforma con la que he tenido el ``placer'' de pelearme. Habría que darle un premio al que hizo la interfaz. Mejor dos, por si va corto.
\item La plataforma en sí está bien, sin más. Creo que para la institución que es la UGR necesita una plataforma mucho más dinámica y moderna (al igual que la plataforma de correo electrónico). Por ejemplo, de forma similar a otras plataformas que he usado en otros centros educativos, podría implementarse como página principal un calendario en el que los profesores puedan añadir tareas, fechas de entrega, calendario de exámenes, y estén disponibles las opciones para los alumnos de ver esos mensajes, tenerlos en forma de calendario para una mejor organización y claridad, y tener la posibilidad de hacer las entregas de aquellos trabajos o documentos que haya que entregar al profesor a partir de la plataforma, en el evento creado en ese mismo calendario por el profesor. También sería interesante la idea de poder usar \texttt{Prado2} a nivel de alumno como plataforma de almacenamiento de documentos, algo así como un ``google drive'' de la universidad. Espero que les sirva de ayuda, esta es mi opinión.
\item La plataforma es un buen apoyo, pero ir aprendiendo todas las funcionalidades requiere tiempo y práctica. Si van a introducir cambios, por favor, que no impliquen reaprenderlo todo de nuevo. Gracias.
\item la utilización de Prado 2 es poco clara e incomoda. No me gusta nada
\item Larga vida a swad.
\item Las interfaces son poco intuitivas. Las funciones parecen que están escondidas, no invitando al usuario a explorar sus funcionalidades.
\item Le falta mucho por mejorar para que sea mucho más práctica
\item Lo dicho arriba
\item Lo siento, pero no, no cuela, prefiero SWAD.
\item mas formacion a los profesores que no tienen ni idea de como funciona
\item Me gustaría poder trabajar más cómodamente con los agrupamientos. Si en una asignatura se trabaja de forma colaborativa en la parte común en lugar de cada profesor en la asignatura correspondiente a su grupo, se genera mucho caos. Por ejemplo, si se quiero mandar un mensaje sólo a mis alumnos, tengo que seleccionarlos uno a uno o crear un foro para mi agrupamiento, cuando se trataba de un simple mensaje. De igual forma, los mensajes en los foros restringidos a los agrupamientos llegan a todos los profesores de la asignatura, lo que genera mucha confusión. Por otra parte, en los libros de calificaciones de los estudiantes aparecen como columnas todas las tareas creadas, incluso si eran para agrupamientos en los que ese alumno no está, lo cual hace que el libro de calificaciones pierda el valor que puede tener para el profesor pues ya no puede comprobar de un vistazo cómo va ese alumno, sino que hay que navegar en una tabla amplísima. Además, resulta difícil importar las calificaciones desde ficheros csv, muchas veces da error aunque el fichero esté bien formado, incluso si está en UTF-8 como se pide.
\item Me gustaría que cada vez que los profesores subieran algo en la plataforma aparecieran en forma de notificaciones, parecido a como era en Swad.
\item Me resulta complicada en general, sobre todo para las calificaciones. Hay diferentes herramientas que no he podido utilizar y que eran mucho más fáciles en swad, creo que deberían poner muchas cosas como en dicha plataforma. Subir archivos era más fácil, poner las evaluaciones y calificaciones también, crear las noticias y eventos próximos para los alumnos también, etc.
\item Mi opinión principal es que la UGR debería de elegir definitivamente 1 sola plataforma de apoyo a la docencias y que solo exista esa.
\item Muchos profesores señalan que se hacen un lío con la plataforma, especialmente a la hora de subir varios archivos a la vez.
\item Muy deficiente
\item Muy necesario que al acceder a la plataforma, nos aparezcan las nuevas notificaciones. No que ahora tenemos que estar accediendo asignatura por asignatura revisando si hay novedades y documentos y mensajes nuevos enviados por profesores.
\item Muy poco accesible, compleja de usar, imposible usar desde dispositivos móviles, no te notifica cuando los profesores suben archivos o calificaciones a la plataforma.
\item Necesita amplias mejoras para ser realmente útil
\item no me resulta cómodo ni rápido
\item No sirve para nada. Los alumnos se quejan constantemente de su ineficiencia. Es necesario subsanar los errores o volver a las plataformas anteriores. Un cordial saludo
\item No soy fanático de ninguna otra plataforma, simplemente, por comodidad y sencillez, Decsai resulta ganadora, en cambio, SWAD, la veo mucho mas ``potente'' y muy completa cuando los profesores saben hacer uso de ella.
\item No sé para qué sirve
\item No veo de mucha utilidad y poco practico tener que ir tema por tema y asignatura por asignatura para tener que saber si los profesores han publicado nuevos archivos. Nadie usa la función de poner las notas de las practicas ni de examenes.
\item Para mi es un sistema adecuado para la enseñanza virtual pero para la presencial es demasiado complejo y se pierde mucho tiempo en cuestiones sencillas
\item Para mi gusto, una auténtica basura. El diseñador tenía nula idea
\item Para ser la herramienta principal de una Universidad le falta mucho por trabajat
\item PD: la encuesta debería ser una (1) encuesta por usuario (1), debido a que hay mucho TROLL por ahí en la universidad de granada (UGR)
\item POR FAVOR, CONVIÉRTANLO EN UNA HERRAMIENTA NO EN UN FIN EN SI MISMO.
\item Por lo demás, la verdad es que la plataforma funciona genial. Un par de necesidades concretas que he tenido (impedir que los alumnos vieran los resultados de unas encuentas, acción que sólo se podía hacer desde el servicio de informática) la verdad es que me han atendido de manera impecable.
\item porque se diseñan plataformas tan poco eficientes? bastaria con poner una sola que sea util , de facil acceso, intuitiva y eficaz, como lo es swad, mejorar lo que hay y no estar todo el dia cambiando de plataformas, aplicaciones,... como si tuvieramos tiempo de estar aprendiendo, modificando, adaptando.... me niego... y por favor, las respuestas de la confeccion de esta encuesta parece que debe ser un programa que da las ordenes en ingles, y yo no tengo porque saber ingles, hay mas programas de encuentas que te an las ordenes en español!!!
\item \texttt{Prado2} es actualmente una plataforma que le queda mucho por desarrollar y también que los profesores den el paso en su aprendizaje. Como posibles mejoras también me gustaría añadir la posibilidad de eliminar los directorios que se encuentran a la derecha y a la izquierda ya que no lo usan prácticamente nadie y no son nada intuitivos.
\item \texttt{Prado2} necesita mejorar, y mucho, el acceso desde dispositivos móviles. Además de que las funciones están demasiado mal organizadas para mi gusto. Buscar funcionalidades en los menús a veces es muy engorroso.
\item preferia tablon de docencia miles de veces antes
\item Prefiero el antiguo Tablón de Docencia al actual sistema de \texttt{Prado2}.
\item Puesto que esta plataforma permite método de evaluación online debería tenerse en cuenta y regularse oficialmente el tiempo límite de realización de la prueba, los intentos, así como el tiempo que permanece disponible al alumnado debido a que pueden surgir distintos fallos (desconexión a la hora de estar realizando la prueba, lentitud del sistema operativo, avería del dispositivo, etc) ya que muchos profesores no lo tienen en cuenta. Debería (en estos casos) haber algún sistema de notificación que avise al profesor en cuestión de que ha ocurrido un fallo durante la realización de la prueba.
\item Que fuese como tablón de docencia era más cómodo y rapido
\item Que le den a Prado. Viva SWAD.
\item Que no se tenga que pasar por diversas pantallas para logarse. Es tedioso.
\item Que quiten el prado 2 y pongan swad de una vez, no se el interes que hay detrás de prado.
\item Que vuelva SWAD, era mucho mas sencilla y estable
\item Quizás la mala experiencia con PRADO va relacionada por el poco provecho que le sacan los profesores, aunque en general, no la veo una página estable para accesos multiplataforma
\item Resulta muy ineficiente, especialmente, para la gestión de grupos de prácticas
\item Sería muy práctico que la página cargase más rápido.
\item Sigo sin entender por que una facultad de Informática utiliza esto, que tiene mucha menos funcionalidad que SWAD o DECSAI incluso.
\item Sinceramente, no sé qué tenía de malo la plataforma SWAD que no se pudiese solucionar y haya sido necesaria la completa migración de la mayoría de la comunidad universitaria a una nueva plataforma docente, con la línea de aprendizaje que ello conlleva y teniendo en cuenta que al usuario estándar no le gusta que le cambien las cosas.
\item Su diseño brilla por su ausencia
\item Suerte con el TFG
\item Suerte!
\item SWAD es mucho mejor, este nuevo sistema es un atraso total
\item Tablón de docencia muchísimo más rápido y fácil de usar. Los profesores lo entendían, con \texttt{Prado2} hay algunos profesores que no saben usarlo y te ves negro para que te pasen los apuntes.
\item También hay asignaturas en las que cada profesor que la imparte tiene su propia 'sección', lo cual es muy lioso.
\item Un sistema que supone una gran pérdida de tiempo para el Profesorado; es un sistema que solo tiene sentido para clases virtuales, pero no para clases presenciales
\item Una plataforma de PRADO2 para Android y iPhone sería bastante útil
\item Una plataforma impuesta a todo un colectivo es de esperar que tenga una mejor usabilidad y debe estar mejor diseñada. En mi caso y en el de otros usuarios de Swad el cambio ha implicado un retroceso.
\item Usad SWAD por Dios. Mucho más rápido, seguro e intuitivo.
\item Usen SWAD
\item Varias veces se queda sin funcionalidad porque muchos alumnos entran a la misma vez.
\item Viva el Betis
\item ¡Qué vuelva SWAD! ;)
\item ¿Hasta qué punto es viable crear una App para PRADO2?
\item ¿Por qué se inutiliza lo que funcionaba y uno había aprendido? ¿Por el copyright? Patético
\item ¿Sería posible que actualizasen la versión de moodle de PRADO para que podamos disfrutar (tanto profesores como alumnos) de las mejoras de versiones posteriores?Esto beneficiaría no sólo por la interfaz sino porque es mucho más fácil hacer uso de las herramientas de moodle
\end{enumerate}




\newpage
\section{Scripts GreaseMonkey}

\subsection{Código fuente Prado HTTPS fixer}

\begin{lstlisting}[language=javascript]
// ==UserScript==
// @name         Prado HTTPS fixer
// @namespace    http://www.ernesto.es/
// @version      0.1
// @description  Make Prado Secure Again!
// @author       Ernesto Serrano
// @match        *://prado.ugr.es/moodle/*
// @grant        none
// @require     js/jquery-ui-1.8.16.custom.min.js
// @resource    customCSS css/ui-darkness/jquery-ui-1.8.16.custom.css
// @require  http://ajax.googleapis.com/ajax/libs/jquery/1.8.3/jquery.min.js
// @require  https://gist.github.com/raw/2625891/waitForKeyElements.js
// ==/UserScript==

// Codigo fix https
if (location.protocol != 'https:')
{
 location.href = 'https:' + window.location.href.substring(window.location.protocol.length);
}

var links,thisLink;
links = document.evaluate("//a[@href]",
    document,
    null,
    XPathResult.UNORDERED_NODE_SNAPSHOT_TYPE,
    null);

for (var i=0;i<links.snapshotLength;i++) {
    var thisLink = links.snapshotItem(i);
    thisLink.href = thisLink.href.replace(RegExp('http://prado\\.ugr\\.es/(.*)'), 'https://prado.ugr.es/$1');
}

for (var el of document.querySelectorAll("img[src]")) {
  el.src = el.src.replace(RegExp('http://prado\\.ugr\\.es/(.*)'),'https://prado.ugr.es/$1');
}

for (var i=0;i<document.styleSheets.length;i++) {
    $('link[href="'+document.styleSheets[i].href+'"]').attr('href', document.styleSheets[i].href.replace(RegExp('http://prado\\.ugr\\.es/(.*)'), 'https://prado.ugr.es/$1'));
}

[].forEach.call( document.querySelectorAll("script[src]"), function( e ) {
    e.src = e.src.replace(RegExp('http://prado\\.ugr\\.es/(.*)'), 'https://prado.ugr.es/$1');
});
// Fin codigo fix https
\end{lstlisting}

\subsection{Código fuente Prado Tuner}
\begin{lstlisting}[language=javascript]
// ==UserScript==
// @name         Prado Tuner
// @namespace    http://www.ernesto.es/
// @version      0.1
// @description  Make Prado Great Again!
// @author       Ernesto Serrano
// @match        *://prado.ugr.es/moodle/*
// @grant        none
// @require     js/jquery-ui-1.8.16.custom.min.js
// @resource    customCSS css/ui-darkness/jquery-ui-1.8.16.custom.css
// @require  http://ajax.googleapis.com/ajax/libs/jquery/1.8.3/jquery.min.js
// @require  https://gist.github.com/raw/2625891/waitForKeyElements.js
// ==/UserScript==

// Desplegamos todos los bloques laterales
$( document ).ready(function() {
     $('.block').css('display', 'block');
});
\end{lstlisting}

\newpage
\section{Scripts en R}
\begin{lstlisting}[language=r]
library(ggplot2)
require(scales)
library(dplyr)
library(tidyr)
library(magrittr)
library(RColorBrewer)
library(ggthemes)
library(zoo)

setwd("/Users/ernesto/Downloads/data_prado")

#mdl_log <- read.csv('log_mod.csv')
mdl_log <- read.csv('log2015.csv')


#campos "ID","TIME","USERID_MD5","IP_MD5","COURSE","COURSE_CODE","TIT","PLAN","CEA","MODULE","CMID","ACTION","URL","INFO"

### Setup additional date variables
mdl_log$time <- as.POSIXlt(mdl_log$TIME, tz = "Europe/Madrid", origin="1970-01-01")
mdl_log$day <- mdl_log$time$mday
mdl_log$month <- mdl_log$time$mon+1 # month of year (zero-indexed)
mdl_log$year <- mdl_log$time$year+1900 # years since 1900
mdl_log$hour <- mdl_log$time$hour
mdl_log$date <- as.Date(mdl_log$time)
mdl_log$week <- format(mdl_log$date, '%Y-%U')
mdl_log$dts <- as.POSIXct(mdl_log$date)
mdl_log$time <- as.POSIXct(mdl_log$time) #la convertimos a POSIXct

mdl_log$dts_str <- interaction(mdl_log$day,mdl_log$month,mdl_log$year,mdl_log$hour,sep='_')
mdl_log$dts_hour <- strptime(mdl_log$dts_str, "%d_%m_%Y_%H")
mdl_log$dts_hour <- as.POSIXct(mdl_log$dts_hour)

#mdl_2015 <- subset(mdl_log, year == 2015)

mdl_2015 <- subset(mdl_log, TIME > 1441058400 & TIME < 1467324000)

mdl_2016 <- subset(mdl_log, TIME > 1441058400 & TIME < 1467324000)

mdl_todo <- subset(mdl_log, TIME > 1441058400 & TIME < 1467324000)


#d <- tbl_df(mdl_log)
d <- tbl_df(mdl_2015)
d <- tbl_df(mdl_2016)
d <- tbl_df(mdl_todo)

d <- tbl_df(mdl_log)
#d %>% mutate(time = as.POSIXct(time))
#d %>% mutate(TIME = as.POSIXct(time))

daily <- group_by(d, USERID_MD5, dts) %>% summarise(Total = n())
daily$dow = as.factor(format(daily$dts, format="%a"))

ggplot(daily, aes(dow, Total)) +
geom_boxplot(aes(fill=dow)) +
scale_x_discrete(limits=c('Mon','Tue','Wed','Thu','Fri','Sat','Sun')) +
theme_few() +
xlab('Dia de la semana') + ylab('Frecuencia de actividad del usuario') +
guides(fill=FALSE)

hourly <- group_by(d, dts_hour) %>% summarise(Total = n())
hourly$dow = as.factor(format(hourly$dts, format="%a"))
hourly$hr = format(hourly$dts_hour, format="%H")
hourly$semana = 'Dia lectivo'
hourly[hourly$dow=='Sat'|hourly$dow=='Sun',]$semana = 'Fin de semana'

cbPalette <- c("#999999", "#E69F00", "#56B4E9", "#009E73",
"#F0E442", "#0072B2", "#D55E00", "#CC79A7")

ggplot(hourly, aes(hr,Total)) +
geom_boxplot(aes(fill=semana)) +
geom_smooth(aes(group=semana)) +
scale_fill_manual(values=cbPalette) +
xlab('Hora del día') + ylab('Frecuencia de actividad diaria')

udaily <- group_by(d, dts) %>% summarise(users = n_distinct(USERID_MD5))

ggplot(udaily, aes(dts, users)) +
geom_bar(stat="identity", fill="#60B3CE") +
scale_x_datetime(breaks = date_breaks("1 week"),
minor_breaks = date_breaks("1 day"),
labels = date_format("%d-%b-%y")) +
scale_color_manual(values=cbPalette) +
theme(axis.text.x = element_text(angle = 90, hjust = 1)) +
labs(x="Día", y="Distinto número de usuarios")

utotal <- group_by(d, USERID_MD5) %>% summarise(Total = n())

ggplot(utotal, aes(reorder(USERID_MD5, Total), Total)) +
geom_point(alpha=0.5, color = "#FF8300") +
scale_x_discrete(breaks=NULL) +
xlab('Usuario') + ylab('Actividad total')

ggplot(utotal, aes(Total)) +
geom_histogram(binwidth=10, fill="#60B3CE") +
xlab('Actividad total') + ylab('Frecuencia de usuarios')

ggplot(actividades2015, aes(y=Total, x=reorder(Modulo,Total))) +geom_bar(stat="identity", width=0.2)  +theme_minimal() + coord_flip() + labs(x="", y="")

\end{lstlisting}

