\chapter{Metodología}

En este capítulo vamos a definir y a detallar los procesos que vamos a seguir para analizar la plataforma en base a los objetivos que presentamos en el capítulo anterior.

\section{Análisis general de la plataforma}
Se realizó un análisis general de la herramienta para conocer enteramente con qué nos encontramos, en qué plataforma está basada y si se han realizado desarrollos adicionales sobre la misma para adecuarlo a las necesidades de la UGR.

\section{Entrevistas personales}

Para recoger la mayor información posible sobre las opiniones de los usuarios de \texttt{Prado2} se realizaron una serie de entrevistas personales de carácter informal con profesores, alumnos y administradores y así poder hacer una valoración global del estado actual de la plataforma, teniendo en cuenta a todos los actores implicados en la misma.

\bigskip
Algunas de estas entrevistas incluyeron un test de usabilidad que consistía en pedir a los usuarios que realizaran una tarea sencilla y ver cuánto tiempo tardaban en realizarla.

\bigskip
Dicha tarea consistía en pedirles cambiar la imagen del perfil de \texttt{Prado2}. El test se basó en el test de usabilidad planteado por Steve Krug en su libro \textit{Don't make me think} \cite{stevekrug} y se pedía que partiendo de una página neutra, como puede ser la pagina de inicio de un buscador, entraran en \texttt{Prado2} y localizaran desde donde podían cambiar su imagen de perfil.


\section{Encuesta a los usuarios}
Uno de los actores más importantes de cualquier plataforma web son sus usuarios por lo que queríamos conocer de primera mano sus opiniones. Para ello se preparó un breve cuestionario online que fue difundido a través de la lista de correo general de la UGR.

\bigskip
La encuesta se realizó con la herramienta \texttt{Google Forms} y estuvo activa desde el 14 de Marzo hasta el 10 de Mayo de 2017, los primeros días se le dio difusión a través de grupos de mensajería y redes sociales. El 21 de Marzo se envió a la lista de correo infoUGR de la Universidad por parte del Tutor y el 23 de Marzo la \texttt{DEIIT} (Delegación de Estudiantes de Ingeniería Informática y Telecomunicación) nos hizo el favor de enviarla a la comunidad de estudiantes de la \texttt{ETSIIT} (Escuela Técnica Superior de Ingeniería Informática y Telecomunicación) ya que la lista infoUGR al estar en un buzón propio no suele tener la difusión que tienen los mensajes que van a la bandeja de entrada.

\bigskip
Dicho cuestionario constaba de varias preguntas preguntas cerradas de respuesta múltiple, ya que las mismas son más objetivas a la hora de hacer un análisis estadístico, también se incluyeron tres preguntas opcionales de respuesta abierta para que los usuarios pudieran exponer razonadamente sus opiniones.

Las preguntas eran las siguientes:

\begin{enumerate}

  \item \textbf{Selección del tipo de usuario} Se pregunta si el usuario es Estudiante, Profesor u otro tipo de miembro de la comunidad universitario, como por ejemplo, doctorandos.

  		\shadowbox{\includegraphics[width=0.9\textwidth]{../screenshots/poll/01}}

  \item \textbf{Selección de la titulación} Se pregunta la titulación del usuario, ya que vimos que las opiniones variaban en base a la misma. El listado oficial de titulaciones lo obtuvimos del Portal de Transparencia de la UGR \url{http://transparente.ugr.es/}

  \shadowbox{\includegraphics[width=0.9\textwidth]{../screenshots/poll/02}}

  \item \textbf{¿Con qué periodicidad accede a \texttt{Prado2}?} Es interesante saber  la periodicidad con la que los usuarios acceden a la plataforma, para además cotejarla con el análisis de los registros de moodle

  \shadowbox{\includegraphics[width=0.9\textwidth]{../screenshots/poll/03}}

  \item \textbf{Califique las siguientes opciones según sus necesidades} Presentamos al usuario una serie de opciones de la plataforma extraídas de las entrevistas personales con varios usuarios, dichas opciones son:

  		\begin{itemize}
  			\item \textbf{Acceso desde dispositivos móviles} La mayor parte de las quejas recibidas antes de empezar este proyecto era la dificultad para acceder desde dispositivos móviles y la falta de una aplicación nativa para dichos dispositivos.
  			\item \textbf{Insignias} Implementación del sistema de insignias de Mozilla Open Badges para otorgar a los alumnos reconocimiento digital de sus cualidades y logros.
  			\item \textbf{Blog de la asignatura} Permitir a las asignaturas mantener un blog dentro de la plataforma, dicho blog solo está accesible para los alumnos de la asignatura.
  			\item \textbf{Blog de los alumnos} Permitir a los alumnos mantener un blog dentro de la plataforma, las publicaciones de dicho blog se pueden configurar para que sean visibles solo para otros alumnos de la misma asignatura o para todos los usuarios de la plataforma.
  			\item \textbf{Mensajería interna} Se debe permite la comunicación entre profesores y alumnos sin salir de la misma.
  			\item \textbf{Estadísticas de uso} Permitir a los alumnos y profesores acceder a las estadísticas de uso de la plataforma.
  			\item \textbf{Calificaciones} Permitir a los alumnos acceder a las calificaciones de sus tareas y exámenes de la asignatura.

		\end{itemize}

		\shadowbox{\includegraphics[width=0.9\textwidth]{../screenshots/poll/04}}


  \item \textbf{Actividades y recursos} Presentamos la lista de actividades que puede crear un profesor en una asignatura para ver cuáles son las más utilizadas (las actividades marcadas con * son actividades desarrolladas expresamente para \texttt{Prado2}).


        \begin{itemize}
            \item \textbf{Auto-selección de Grupo *} Esta actividad permite a los alumnos seleccionar su grupo de prácticas al inicio del curso.

            \item \textbf{Base de datos}: Les permite a los participantes crear, mantener y buscar dentro de un banco de entradas de registros.

            \item \textbf{Chat}: Les permite a los participantes tener una discusión en tiempo real.

            \item \textbf{Consulta}: Un profesor hace una pregunta y especifica una variedad de respuestas de opción múltiple.

            \item \textbf{Control de Asistencia *}: Esta actividad permite controlar la asistencia a clase.

            \item \textbf{Cuestionario}: Permite al profesor diseñar exámenes, que pueden ser calificados automáticamente.

            \item \textbf{Encuestas predefinidas}: Para recolectar datos de los estudiantes, para ayudar a los profesores a conocer sus alumnos y reflexionar sobre su enseñanza.

            \item \textbf{Foro}: Permite a los participantes tener discusiones asíncronas.

            \item \textbf{Glosario}: Permite a los participantes crear y mantener una lista de definiciones, a semejanza de un diccionario.

            \item \textbf{Herramienta Externa}: Permite a los participantes interactuar con recursos y actividades de enseñanza compatibles con LTI\footnote{Learning Tools Interoperability (LTI) es una especificación desarrollada por IMS Global Learning Consortium.} en otros sitios web.

            \item \textbf{Paquete SCORM}: Permite que se incluyan paquetes SCORM\footnote{SCORM (del inglés Sharable Content Object Reference Model) es un conjunto de estándares y especificaciones que permite crear objetos pedagógicos estructurados.} como contenido del curso.

            \item \textbf{Lección}: Para proporcionar contenido en formas flexibles.

            \item \textbf{Taller}: Habilita la evaluación por pares.

            \item \textbf{Tarea}: Permite a los profesores calificar y hacer comentarios sobre archivos subidos y tareas creadas en línea y fuera de línea

            \item \textbf{Wiki}: Una colección de páginas web donde cualquiera puede añadir o editar.

            \item \textbf{Archivo}: Son materiales que podemos poner a disposición del alumnado en diferentes formatos: documentos de texto, imágenes, vídeos, audios, archivos comprimidos, etc.

            \item \textbf{Carpeta}: Se utilizan para poner a disposición del alumnado múltiples archivos agrupados en un ítem.

            \item \textbf{Clase grabada (GA3) *}: Esta actividad permite subir una clase grabada con el sistema GA3 para su posterior visualización.

            \item \textbf{Etiqueta}: Se utilizan para insertar pequeñas secciones de texto, imágenes o elementos multimedia entre los distintos bloques de contenido del curso.

            \item \textbf{Libro}: Recursos multi-página con aspecto similar a un libro. Los profesores pueden exportar sus Libros como paquete IMS (el administrador debe permitir que el rol de profesor pueda exportar IMS)

            \item \textbf{Página}: Permiten añadir contenido directamente en \texttt{moodle} mediante el editor de texto disponible.

            \item \textbf{Paquete de contenido IMS}: Es un tipo de formato de archivo estándar basado en una serie de especificaciones que facilitan la reutilización de contenidos en distintos sistemas sin necesidad de convertirlos a otro formato.

            \item \textbf{URL}: Permite agregar un enlace a un sitio web.

        \end{itemize}
		\shadowbox{\includegraphics[width=0.9\textwidth]{../screenshots/poll/05}}


  \item \textbf{¿Cómo calificaría \texttt{Prado2}?} Presentamos al usuario los aspectos mas importantes para calificar la plataforma, dichos aspectos son:

  		\begin{itemize}
  			\item \textbf{Usabilidad} Define si un sistema es sencillo de usar, facilita la lectura, presenta funciones y menús sencillos, por lo que es cómodo su uso.
  			\item \textbf{Accesibilidad} Grado en el que todas las personas pueden utilizar el servicio, independientemente de sus capacidades técnicas, cognitivas o físicas.
  			\item \textbf{Seguridad} Define la sensación de robustez de un sistema frente a ataques informáticos.
  			\item \textbf{Disponibilidad} Disponibilidad: Medida que indica cuanto tiempo está disponible el sistema y lo resistente que es frente a caídas.
  			\item \textbf{Valoración General} La valoración de la plataforma en su conjunto.

		\end{itemize}

    \shadowbox{\includegraphics[width=0.9\textwidth]{../screenshots/poll/06}}

  \item \textbf{¿Con cuales de las siguientes plataformas de la UGR ha trabajado?} Presentamos al usuario una lista con las plataformas de docencia más importantes de la UGR:

  		\begin{itemize}
  			\item DECSAI
            \item Tutor
            \item SWAD
            \item Tablón de docencia
            \item Prado 1 CEVUG (moodle)
            \item Prado 2 (moodle)
            \item Otro
		\end{itemize}

  \shadowbox{\includegraphics[width=0.9\textwidth]{../screenshots/poll/07}}

  \item \textbf{De las siguientes plataformas de la UGR ¿Cuál prefiere?} Presentamos al usuario una lista con las plataformas de docencia más importantes de la UGR:

  		\begin{itemize}
  			\item DECSAI
            \item Tutor
            \item SWAD
            \item Tablón de docencia
            \item Prado 1 -c(moodle)
            \item Prado 2 (moodle)
            \item Otro
		\end{itemize}

    \shadowbox{\includegraphics[width=0.9\textwidth]{../screenshots/poll/08}}

  \item \textbf{¿Ha encontrado algún error que debería ser subsanado en \texttt{Prado2}?} Permitimos al usuario expresar su opinión detallada.

  \shadowbox{\includegraphics[width=0.9\textwidth]{../screenshots/poll/09}}


  \item \textbf{Indique si hay alguna función adicional que le gustaría ver en \texttt{Prado2}} Permitimos al usuario expresar su opinión detallada.

  \shadowbox{\includegraphics[width=0.9\textwidth]{../screenshots/poll/10}}

  \item \textbf{Indique cuál es la funcionalidad que más le gusta de \texttt{Prado2}} Permitimos al usuario expresar su opinión detallada.

  \shadowbox{\includegraphics[width=0.9\textwidth]{../screenshots/poll/11}}

  \item \textbf{E-mail} Permitimos al usuario proporcionarnos su dirección de e-mail para futuros contactos.

  \shadowbox{\includegraphics[width=0.9\textwidth]{../screenshots/poll/12}}

  \item \textbf{Comentarios adicionales} Permitimos al usuario expresar su opinión detallada

  \shadowbox{\includegraphics[width=0.9\textwidth]{../screenshots/poll/13}}

\end{enumerate}

\section{Análisis de registros propios de la plataforma}

La plataforma \texttt{moodle} provee de mecanismos para registrar el uso de la misma en diversas tablas de registro, particularmente la versión utilizada usa la tabla \texttt{mdl\_log} (figura \ref{logtable}) para almacenar los registros de acceso de los usuarios. Cabe indicar que a partir de la versión 2.7 de \texttt{moodle} la tabla\texttt{mdl\_log}  ya no se utiliza y ha pasado a marcarse como obsoleta\cite{art_09} en favor de la tabla \texttt{mdl\_logstore\_standard\_log}.

Dicha tabla mantiene un registro de todas las interacciones de los usuarios en la página así que podemos extraer información interesante como el tiempo medio de la plataforma por parte de los usuarios o que actividades tienen mas uso.

Como la tabla contenía información personal como puede ser el identificador del usuario y su dirección ip se optó por diseñar una consulta SQL que retornara los datos anonimizando los datos sensibles mediante una función hash.

\begin{figure}\centering\includegraphics[width=0.8\textwidth]{../images/logtable}\caption{Tabla de log}\label{logtable}\end{figure}

\begin{lstlisting}[language=sql]
SELECT
    id, time, MD5(userid), MD5(ip), MD5(course), module, cmid, action, url, info
FROM
    mdl_log
\end{lstlisting}


Tras una reunión en el \texttt{CEVUG} con los responsables de \texttt{Prado2} mantenida el 23 de Marzo de 2017 donde nos transmitieron su apoyo y colaboración acordamos realizar una petición formal de datos mediante una solicitud a la Secretaria General de la Universidad de Granada.

Una vez definida la consulta se intentó realizar dicha petición formal de datos. Tras varios intentos infructuosos de realizar la petición de forma telemática por parte del tutor se optó por realizarla de forma presencial en papel a través del registro de la Universidad de Granada.

El día 23 de Mayo nos remitieron desde el CEVUG una muestra de los datos para ver si los datos eran correctos antes de darnos en fases los datos completos. La primera fase correspondiente a los datos del curso 2015/2016 los tuvimos disponibles el 1 de Junio en un fichero CSV\footnote{Los archivos CSV (del inglés comma-separated values) son un tipo de documento para representar datos en forma de tabla en las que las columnas se separan por comas.} de 500 megabytes conteniendo más de 3 millones de registros. El 7 de junio de 2017 recibimos los datos de 2016/2017 en un fichero de 3,79 gigabytes conteniendo más de 20 millones de registros, dichos datos solo contemplaban los registros hasta el 31 de Diciembre de 2016.

\begin{lstlisting}[language=sql]
SQL> select  /*csv*/ l.id,l.time,rawtohex(sys.DBMS_OBFUSCATION_TOOLKIT.MD5(INPUT_STRING => l.userid)) userid_md5,rawtohex(sys.DBMS_OBFUSCATION_TOOLKIT.MD5(INPUT_STRING => l.ip)) ip_md5 ,l.course, substr(c.idnumber,0,14) course_code,substr(c.idnumber,0,3) tit, substr(c.idnumber,5,2) plan, substr(c.idnumber,8,2) cea, l.module,l.cmid, l.action, l.url, l.info from p_log l, p_course c where (l.course=c.id and l.userid>10 and REGEXP_LIKE(c.idnumber, '^[0-9]') and  l.time>1420070400 and l.time<1451606400);

\end{lstlisting}

Con los datos recogidos y basándonos en el excelente articulo\cite{art_02} de James Ballard, que aunque se centra en en el análisis de la nueva tabla de log de las versiones 2.7 y superiores, escribimos una serie de consultas en lenguaje \texttt{R} que procesara los datos para sacar conclusiones de los mismo. Dichas consultas se adjuntan en el anexo.

\section{Análisis de usabilidad de la plataforma}

Para el análisis de usabilidad hicimos uso de las directrices del experto en usabilidad Jakob Nielsen \cite{jakonielsen} conocidas como ``Heurísticas de Nielsen'' para crear sistemas que sean amigables para el usuario:

\begin{enumerate}
\item \textbf{Visibilidad del estado del sistema.} El sistema debe informar a los usuarios del estado del sistema, dando una retroalimentación apropiada en un tiempo razonable.
\item \textbf{Utilizar el lenguaje de los usuarios.} El sistema debe utilizar el lenguaje de los usuarios, con palabras o frases que le sean conocidas, en lugar de los términos que se utilizan en el sistema, para que al usuario no se le dificulte utilizar el sistema.
\item \textbf{Control y libertad para el usuario.} En casos en los que los usuarios elijan una opción del sistema por error, éste debe contar con las opciones de deshacer y rehacer para proveer al usuario de una salida fácil sin tener que utilizar diálogo extendido.
\item \textbf{Consistencia y estándares.} El usuario debe seguir las normas y convenciones de la plataforma sobre la que está implementando el sistema, para que no se tenga que preguntar el significado de las palabras, situaciones o acciones del sistema.
\item \textbf{Prevención de errores.} Es más importante prevenir la aparición de errores que generar buenos mensajes de error. Hay que eliminar acciones predispuestas al error o, en todo caso, localizarlas y preguntar al usuario si está seguro de realizarlas.
\item \textbf{Minimizar la carga de la memoria del usuario.} El sistema debe minimizar la información que el usuario debe recordar mostrándola a través de objetos, acciones u opciones. El usuario no tiene por qué recordar la información que recibió anteriormente. Las instrucciones para el uso del sistema deberían ser visibles o estar al alcance del usuario cuando se requieran.
\item \textbf{Flexibilidad y eficiencia de uso.} Los aceleradores permiten aumentar la velocidad de interacción para el usuario experto tal que el sistema pueda atraer a usuarios principiantes y experimentados. Es importante que el sistema permita personalizar acciones frecuentes para así acelerar el uso de éste.
\item \textbf{Diálogos estéticos y diseño minimalista.} La interfaz no debe contener información que no sea relevante o se utilice raramente, pues cada unidad adicional de información en un diálogo compite con las unidades relevantes de la información y disminuye su visibilidad relativa.
\item \textbf{Ayudar a los usuarios a reconocer, diagnosticar y recuperarse de los errores.} Los mensajes de error deben expresarse en un lenguaje claro, indicar exactamente el problema y ser constructivos.
\item \textbf{Ayuda y documentación.} A pesar de que es mejor un sistema que no necesite documentación, puede ser necesario disponer de ésta. Si así es, la documentación tiene que ser fácil de encontrar, estar centrada en las tareas del usuario, tener información de las etapas a realizar y no ser muy extensa.
\end{enumerate}


Además se tuvo en consideración las recomendaciones de usabilidad de María Elena Alva\cite{melenaalva} y Eric Reiss\cite{ericreiss}, así como mi experiencia de varios años trabajando como desarrollador web.

\bigskip
Siguiendo todas estas recomendaciones se comprobó cada uno los problemas indicados en las entrevistas personales, intentando reproducirlos. También se comprobaron otros detalles de usabilidad no mencionados en las entrevistas.

\section{Análisis de accesibilidad de la plataforma}

La accesibilidad para personas con alguna discapacidad se analizó usando diversas herramientas online que validan si una plataforma cumple las normativas de accesibilidad.

\bigskip
La legislación Española exige a páginas de organismos públicos el nivel de adecuación \texttt{AA} de la Norma \texttt{UNE 139803}, que en 2012 se actualizó para ser equivalente a las \texttt{WCAG 2.0}.

\bigskip
Las herramientas utilizadas han sido el test TAW\cite{taw}, la web Tenon.io\cite{tenon} y HTML CodeSniffer\cite{codesniffer}.

\bigskip
Adicionalmente ampliamos el análisis para comprobar además la accesibilidad desde dispositivos móviles y el soporte multi-idioma de Prado.

\section{Análisis de seguridad}

Para analizar la seguridad de la plataforma se realizó una auditoría para ver que información sensible se podía extraer de la misma. Para la misma se analizaron las vulnerabilidades específicas de la versión instalada mirando en distintos repositorios de exploits así como en el propio listado de vulnerabilidades (CVE) de \texttt{moodle} que se puede encontrar en \url{https://moodle.org/security/}.

\bigskip
\texttt{Prado2} no tiene activado por defecto el protocolo \texttt{https} aunque si tengan un certificado \texttt{ssl} instalado, por lo que la página podía ser vulnerable a secuestros de sesión \texttt{(session hijacking)}. Para comprobar si esto era posible y haciendo uso de la herramienta \textbf{ettercap} lanzamos un ataque de ARP Spoofing\cite{art_08} que hiciera que todas las peticiones \texttt{http} que pasaran por la red pasaran por nuestra máquina que estaría capturando los datos, los ataques de este tipo son bien conocidos desde 1997\cite{art_07} por lo que los propios desarrolladores de \texttt{moodle} recomiendan habilitar por defecto el protocolo \texttt{https}\cite{art_10}.

\bigskip
Para ejecutar dicho ataque solo tuvimos que llamar a \texttt{ettercap} con una serie de parámetros:

\begin{lstlisting}
sudo ettercap -T -i en0 -P autoadd -M arp ///80 -e "^GET /moodle/ HTTP/1.[01].*MOODLEID1_=" -t tcp
\end{lstlisting}

Para optimizar el ataque incluimos como parámetro una expresión regular que se centra en obtener la variable de sesión de \texttt{moodle} para así reducir el tamaño de las capturas. Si la plataforma es vulnerable podremos ver las variables de sesión \textbf{MOODLEID1\_}, \textbf{SimpleSAMLSessionID-sp} y \textbf{SimpleSAMLAuthToken} en cuanto un usuario se identifique en la página.

\bigskip
Como no queríamos causar ningún tipo de problema al resto de usuarios decidimos lanzar un ataque dirigido en el que se intentó robar las variables de sesión de \texttt{moodle} del tutor bajo su consentimiento expreso.

\section{Análisis de disponibilidad}

Para el análisis de  disponibilidad de la plataforma se optó por monitorizar la web mediante la herramienta \texttt{StatusCake} \cite{statuscake}.

\bigskip
Dicha herramienta va lanzando continuamente peticiones de conexión a la página desde diferentes centros de datos a lo largo del mundo y va registrando el tiempo que tarda en responder y el tipo de respuesta para saber si la página está operativa o caída.

\bigskip
Uno de los aspectos interesantes de esta herramienta es que se pueden configurar varios tipos de avisos por lo que si la plataforma sufre una interrupción del servicio nos lo notifica por e-mail, avisando también cuando vuelve a estar operativo.



\section{Clonado de la plataforma}

Con la información obtenida en en análisis general se planteó la posibilidad de poner un funcionamiento un clon de la plataforma en un servidor propio, en un primer contacto los administradores nos dijeron que lo mismo no sería posible por lo que se optó por realizar una instalación de la misma versión de \texttt{moodle} que hay actualmente en \texttt{Prado2} así como la plantilla utilizada haciendo las mínimas adaptaciones necesarias para que fuera lo más parecido posible a la instalación actual.

\section{Realización de scripts GreaseMonkey}

Para comprobar si sería posible solucionar parcialmente los problemas de seguridad y accesibilidad sin requerir la intervención de los administradores de Prado se planteó realizar una serie de scripts en formato \texttt{GreaseMonkey} que mediante \texttt{Javascript} pudiera hacer los cambios mínimos para:

  		\begin{enumerate}
  			\item Forzar que la página se sirva mediante el protocolo \texttt{https}.
            \item Realizar simples cambios estéticos para mejorar la usabilidad.
        \end{enumerate}

\section{Herramienta hardware para robo de sesiones}

A modo de ejercicio práctico para ver el alcance de la seguridad de la plataforma se propuso crear un dispositivo hardware para cazar credenciales de usuarios a modo de \texttt{honeypot} utilizando un viejo router instalando el firmware \texttt{OpenWRT} \cite{openwrt} e instalando las herramientas necesarias para obtener los identificadores de sesión de los clientes conectados.

\bigskip
Mediante esta técnica un posible atacante podría ir a un lugar con gran afluencia de alumnos y/o profesores para secuestrar de forma sencilla sus sesiones de \texttt{Prado2}.



