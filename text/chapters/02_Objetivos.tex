\chapter{Objetivos y Temporización}


El objetivo máximo de este proyecto es realizar un análisis detallado de la plataforma \texttt{Prado2} que pueda servir como una herramienta para facilitar el trabajo a los administradores de la misma.

\bigskip
Dicho objetivo se descompone los siguientes objetivos principales:

\begin{itemize}
  \item \textbf{OBJ-1.} Analizar la usabilidad de \texttt{Prado2}.
  \item \textbf{OBJ-2.} Analizar la accesibilidad de \texttt{Prado2}.
  \item \textbf{OBJ-3.} Analizar la seguridad de \texttt{Prado2}.
  \item \textbf{OBJ-4.} Analizar la disponibilidad de \texttt{Prado2}.
  \item \textbf{OBJ-5.} Promover la idea de que los desarrollos bajo software libre en general y en la administración pública en particular ayudan a generar confianza y demuestran compromiso.

\end{itemize}

Además como objetivos secundarios tendremos:

\begin{itemize}
  \item \textbf{OBJ-6.} Estudiar la posibilidad de hacer un clon de la instalación actual de \texttt{Prado2} en un servidor propio.
  \item \textbf{OBJ-7.} Estudiar la posibilidad de desarrollar una herramienta para solventar los problemas de usabilidad de la plataforma.
  \item \textbf{OBJ-8.} Estudiar la posibilidad de desarrollar una herramienta para solventar los problemas de seguridad de la plataforma.
   \item \textbf{OBJ-9.} Estudiar la posibilidad de configurar un dispositivo hardware para analizar la seguridad de la plataforma.
\end{itemize}



\section{Alcance de los objetivos}
El fin inmediato de este informe es servir de herramienta a los administradores de la plataforma para que conozcan los fallos más importantes y sugerencias sobre como solucionarlos.
Además el informe resultante se liberará con una licencia libre para que cualquier parte interesada pueda hacer uso de las conclusiones y los datos extraídos del análisis.

\section{Interdependencia de los objetivos}

Todos los objetivos son independientes entre sí, pero el primer objetivo (\textbf{OBJ-1}) es el principal motivador de este proyecto, por lo que aún sin representar el desarrollo de ningún trabajo en concreto es el que va a escudar y avalar el desarrollo de los otros. En aspectos más relacionados con la realización del proyecto, el tercer objetivo (\textbf{OBJ-3}) es el que se requerirán soluciones de forma inmediata, ya que impide el correcto funcionamiento del portal. El resto de objetivos secundarios, al no tener un carácter urgente serán resueltos en base a la disponibilidad del tiempo necesario para su realización.

\section{Conocimientos y herramientas utilizadas}

\bigskip
Destacar en los aspectos formativos previos más utilizados para el desarrollo del proyecto los conocimientos adquiridos en las asignaturas ``Desarrollo de Aplicaciones para Internet'' para el análisis de usabilidad y accesibilidad, ``Fundamentos de Ingeniería de Software'' para el análisis del proyecto, ``Seguridad en Sistemas Operativos'' para la parte de seguridad y ``Servidores Web de Altas Prestaciones'' para la realización de pruebas desde el punto de vista de disponibilidad y carga de trabajo.

Para la realización de cada una de las partes se han usado multitud de herramientas específicas tales como pueden ser \texttt{WireShark}, \texttt{ettercap}, \texttt{statusCake}, \texttt{Latex}, \texttt{R}, \texttt{Google Forms} y \texttt{Google SpreadSheets} entre otras.


\section{Temporización estimada}


En la figura \ref{fig:temporizacion1} podemos ver la planificación de tiempos para el desarrollo de este proyecto.

\begin{figure}[H]
\centering
\includegraphics[width=1.3\textwidth,angle=90]{../screenshots/temporizacion1}
\caption{Diagrama de Gantt con los tiempos estimados para el proyecto}
\label{fig:temporizacion1}
\end{figure}







